% Generated by Sphinx.
\def\sphinxdocclass{report}
\documentclass[letterpaper,10pt,english]{sphinxmanual}
\usepackage[utf8]{inputenc}
\DeclareUnicodeCharacter{00A0}{\nobreakspace}
\usepackage{cmap}
\usepackage[T1]{fontenc}
\usepackage{babel}
\usepackage{times}
\usepackage[Bjarne]{fncychap}
\usepackage{longtable}
\usepackage{sphinx}
\usepackage{multirow}


\title{PV\_LIB for Python Documentation}
\date{May 15, 2014}
\release{1.0}
\author{Sandia National Labs}
\newcommand{\sphinxlogo}{}
\renewcommand{\releasename}{Release}
\makeindex

\makeatletter
\def\PYG@reset{\let\PYG@it=\relax \let\PYG@bf=\relax%
    \let\PYG@ul=\relax \let\PYG@tc=\relax%
    \let\PYG@bc=\relax \let\PYG@ff=\relax}
\def\PYG@tok#1{\csname PYG@tok@#1\endcsname}
\def\PYG@toks#1+{\ifx\relax#1\empty\else%
    \PYG@tok{#1}\expandafter\PYG@toks\fi}
\def\PYG@do#1{\PYG@bc{\PYG@tc{\PYG@ul{%
    \PYG@it{\PYG@bf{\PYG@ff{#1}}}}}}}
\def\PYG#1#2{\PYG@reset\PYG@toks#1+\relax+\PYG@do{#2}}

\expandafter\def\csname PYG@tok@gd\endcsname{\def\PYG@tc##1{\textcolor[rgb]{0.63,0.00,0.00}{##1}}}
\expandafter\def\csname PYG@tok@gu\endcsname{\let\PYG@bf=\textbf\def\PYG@tc##1{\textcolor[rgb]{0.50,0.00,0.50}{##1}}}
\expandafter\def\csname PYG@tok@gt\endcsname{\def\PYG@tc##1{\textcolor[rgb]{0.00,0.27,0.87}{##1}}}
\expandafter\def\csname PYG@tok@gs\endcsname{\let\PYG@bf=\textbf}
\expandafter\def\csname PYG@tok@gr\endcsname{\def\PYG@tc##1{\textcolor[rgb]{1.00,0.00,0.00}{##1}}}
\expandafter\def\csname PYG@tok@cm\endcsname{\let\PYG@it=\textit\def\PYG@tc##1{\textcolor[rgb]{0.25,0.50,0.56}{##1}}}
\expandafter\def\csname PYG@tok@vg\endcsname{\def\PYG@tc##1{\textcolor[rgb]{0.73,0.38,0.84}{##1}}}
\expandafter\def\csname PYG@tok@m\endcsname{\def\PYG@tc##1{\textcolor[rgb]{0.13,0.50,0.31}{##1}}}
\expandafter\def\csname PYG@tok@mh\endcsname{\def\PYG@tc##1{\textcolor[rgb]{0.13,0.50,0.31}{##1}}}
\expandafter\def\csname PYG@tok@cs\endcsname{\def\PYG@tc##1{\textcolor[rgb]{0.25,0.50,0.56}{##1}}\def\PYG@bc##1{\setlength{\fboxsep}{0pt}\colorbox[rgb]{1.00,0.94,0.94}{\strut ##1}}}
\expandafter\def\csname PYG@tok@ge\endcsname{\let\PYG@it=\textit}
\expandafter\def\csname PYG@tok@vc\endcsname{\def\PYG@tc##1{\textcolor[rgb]{0.73,0.38,0.84}{##1}}}
\expandafter\def\csname PYG@tok@il\endcsname{\def\PYG@tc##1{\textcolor[rgb]{0.13,0.50,0.31}{##1}}}
\expandafter\def\csname PYG@tok@go\endcsname{\def\PYG@tc##1{\textcolor[rgb]{0.20,0.20,0.20}{##1}}}
\expandafter\def\csname PYG@tok@cp\endcsname{\def\PYG@tc##1{\textcolor[rgb]{0.00,0.44,0.13}{##1}}}
\expandafter\def\csname PYG@tok@gi\endcsname{\def\PYG@tc##1{\textcolor[rgb]{0.00,0.63,0.00}{##1}}}
\expandafter\def\csname PYG@tok@gh\endcsname{\let\PYG@bf=\textbf\def\PYG@tc##1{\textcolor[rgb]{0.00,0.00,0.50}{##1}}}
\expandafter\def\csname PYG@tok@ni\endcsname{\let\PYG@bf=\textbf\def\PYG@tc##1{\textcolor[rgb]{0.84,0.33,0.22}{##1}}}
\expandafter\def\csname PYG@tok@nl\endcsname{\let\PYG@bf=\textbf\def\PYG@tc##1{\textcolor[rgb]{0.00,0.13,0.44}{##1}}}
\expandafter\def\csname PYG@tok@nn\endcsname{\let\PYG@bf=\textbf\def\PYG@tc##1{\textcolor[rgb]{0.05,0.52,0.71}{##1}}}
\expandafter\def\csname PYG@tok@no\endcsname{\def\PYG@tc##1{\textcolor[rgb]{0.38,0.68,0.84}{##1}}}
\expandafter\def\csname PYG@tok@na\endcsname{\def\PYG@tc##1{\textcolor[rgb]{0.25,0.44,0.63}{##1}}}
\expandafter\def\csname PYG@tok@nb\endcsname{\def\PYG@tc##1{\textcolor[rgb]{0.00,0.44,0.13}{##1}}}
\expandafter\def\csname PYG@tok@nc\endcsname{\let\PYG@bf=\textbf\def\PYG@tc##1{\textcolor[rgb]{0.05,0.52,0.71}{##1}}}
\expandafter\def\csname PYG@tok@nd\endcsname{\let\PYG@bf=\textbf\def\PYG@tc##1{\textcolor[rgb]{0.33,0.33,0.33}{##1}}}
\expandafter\def\csname PYG@tok@ne\endcsname{\def\PYG@tc##1{\textcolor[rgb]{0.00,0.44,0.13}{##1}}}
\expandafter\def\csname PYG@tok@nf\endcsname{\def\PYG@tc##1{\textcolor[rgb]{0.02,0.16,0.49}{##1}}}
\expandafter\def\csname PYG@tok@si\endcsname{\let\PYG@it=\textit\def\PYG@tc##1{\textcolor[rgb]{0.44,0.63,0.82}{##1}}}
\expandafter\def\csname PYG@tok@s2\endcsname{\def\PYG@tc##1{\textcolor[rgb]{0.25,0.44,0.63}{##1}}}
\expandafter\def\csname PYG@tok@vi\endcsname{\def\PYG@tc##1{\textcolor[rgb]{0.73,0.38,0.84}{##1}}}
\expandafter\def\csname PYG@tok@nt\endcsname{\let\PYG@bf=\textbf\def\PYG@tc##1{\textcolor[rgb]{0.02,0.16,0.45}{##1}}}
\expandafter\def\csname PYG@tok@nv\endcsname{\def\PYG@tc##1{\textcolor[rgb]{0.73,0.38,0.84}{##1}}}
\expandafter\def\csname PYG@tok@s1\endcsname{\def\PYG@tc##1{\textcolor[rgb]{0.25,0.44,0.63}{##1}}}
\expandafter\def\csname PYG@tok@gp\endcsname{\let\PYG@bf=\textbf\def\PYG@tc##1{\textcolor[rgb]{0.78,0.36,0.04}{##1}}}
\expandafter\def\csname PYG@tok@sh\endcsname{\def\PYG@tc##1{\textcolor[rgb]{0.25,0.44,0.63}{##1}}}
\expandafter\def\csname PYG@tok@ow\endcsname{\let\PYG@bf=\textbf\def\PYG@tc##1{\textcolor[rgb]{0.00,0.44,0.13}{##1}}}
\expandafter\def\csname PYG@tok@sx\endcsname{\def\PYG@tc##1{\textcolor[rgb]{0.78,0.36,0.04}{##1}}}
\expandafter\def\csname PYG@tok@bp\endcsname{\def\PYG@tc##1{\textcolor[rgb]{0.00,0.44,0.13}{##1}}}
\expandafter\def\csname PYG@tok@c1\endcsname{\let\PYG@it=\textit\def\PYG@tc##1{\textcolor[rgb]{0.25,0.50,0.56}{##1}}}
\expandafter\def\csname PYG@tok@kc\endcsname{\let\PYG@bf=\textbf\def\PYG@tc##1{\textcolor[rgb]{0.00,0.44,0.13}{##1}}}
\expandafter\def\csname PYG@tok@c\endcsname{\let\PYG@it=\textit\def\PYG@tc##1{\textcolor[rgb]{0.25,0.50,0.56}{##1}}}
\expandafter\def\csname PYG@tok@mf\endcsname{\def\PYG@tc##1{\textcolor[rgb]{0.13,0.50,0.31}{##1}}}
\expandafter\def\csname PYG@tok@err\endcsname{\def\PYG@bc##1{\setlength{\fboxsep}{0pt}\fcolorbox[rgb]{1.00,0.00,0.00}{1,1,1}{\strut ##1}}}
\expandafter\def\csname PYG@tok@kd\endcsname{\let\PYG@bf=\textbf\def\PYG@tc##1{\textcolor[rgb]{0.00,0.44,0.13}{##1}}}
\expandafter\def\csname PYG@tok@ss\endcsname{\def\PYG@tc##1{\textcolor[rgb]{0.32,0.47,0.09}{##1}}}
\expandafter\def\csname PYG@tok@sr\endcsname{\def\PYG@tc##1{\textcolor[rgb]{0.14,0.33,0.53}{##1}}}
\expandafter\def\csname PYG@tok@mo\endcsname{\def\PYG@tc##1{\textcolor[rgb]{0.13,0.50,0.31}{##1}}}
\expandafter\def\csname PYG@tok@mi\endcsname{\def\PYG@tc##1{\textcolor[rgb]{0.13,0.50,0.31}{##1}}}
\expandafter\def\csname PYG@tok@kn\endcsname{\let\PYG@bf=\textbf\def\PYG@tc##1{\textcolor[rgb]{0.00,0.44,0.13}{##1}}}
\expandafter\def\csname PYG@tok@o\endcsname{\def\PYG@tc##1{\textcolor[rgb]{0.40,0.40,0.40}{##1}}}
\expandafter\def\csname PYG@tok@kr\endcsname{\let\PYG@bf=\textbf\def\PYG@tc##1{\textcolor[rgb]{0.00,0.44,0.13}{##1}}}
\expandafter\def\csname PYG@tok@s\endcsname{\def\PYG@tc##1{\textcolor[rgb]{0.25,0.44,0.63}{##1}}}
\expandafter\def\csname PYG@tok@kp\endcsname{\def\PYG@tc##1{\textcolor[rgb]{0.00,0.44,0.13}{##1}}}
\expandafter\def\csname PYG@tok@w\endcsname{\def\PYG@tc##1{\textcolor[rgb]{0.73,0.73,0.73}{##1}}}
\expandafter\def\csname PYG@tok@kt\endcsname{\def\PYG@tc##1{\textcolor[rgb]{0.56,0.13,0.00}{##1}}}
\expandafter\def\csname PYG@tok@sc\endcsname{\def\PYG@tc##1{\textcolor[rgb]{0.25,0.44,0.63}{##1}}}
\expandafter\def\csname PYG@tok@sb\endcsname{\def\PYG@tc##1{\textcolor[rgb]{0.25,0.44,0.63}{##1}}}
\expandafter\def\csname PYG@tok@k\endcsname{\let\PYG@bf=\textbf\def\PYG@tc##1{\textcolor[rgb]{0.00,0.44,0.13}{##1}}}
\expandafter\def\csname PYG@tok@se\endcsname{\let\PYG@bf=\textbf\def\PYG@tc##1{\textcolor[rgb]{0.25,0.44,0.63}{##1}}}
\expandafter\def\csname PYG@tok@sd\endcsname{\let\PYG@it=\textit\def\PYG@tc##1{\textcolor[rgb]{0.25,0.44,0.63}{##1}}}

\def\PYGZbs{\char`\\}
\def\PYGZus{\char`\_}
\def\PYGZob{\char`\{}
\def\PYGZcb{\char`\}}
\def\PYGZca{\char`\^}
\def\PYGZam{\char`\&}
\def\PYGZlt{\char`\<}
\def\PYGZgt{\char`\>}
\def\PYGZsh{\char`\#}
\def\PYGZpc{\char`\%}
\def\PYGZdl{\char`\$}
\def\PYGZhy{\char`\-}
\def\PYGZsq{\char`\'}
\def\PYGZdq{\char`\"}
\def\PYGZti{\char`\~}
% for compatibility with earlier versions
\def\PYGZat{@}
\def\PYGZlb{[}
\def\PYGZrb{]}
\makeatother

\begin{document}

\maketitle
\tableofcontents
\phantomsection\label{index::doc}

\begin{quote}

Test Script 1
\end{quote}


\chapter{Irradiance and atmospheric  functions}
\label{index:irradiance-and-atmosperhic-functions}\label{index:welcome-to-pv-lib-s-documentation}
\begin{longtable}{lp{7cm}}
\hline
\endfirsthead

\multicolumn{2}{c}%
{{\textsf{\tablename\ \thetable{} -- continued from previous page}}} \\
\hline
\endhead

\hline \multicolumn{2}{|r|}{{\textsf{Continued on next page}}} \\ \hline
\endfoot

\endlastfoot


{\hyperref[stubs/pvlib.pvl_alt2pres:pvlib.pvl_alt2pres]{\code{pvlib.pvl\_alt2pres}}}(altitude)
 & 
Determine site pressure from altitude
\\

{\hyperref[stubs/pvlib.pvl_pres2alt:pvlib.pvl_pres2alt]{\code{pvlib.pvl\_pres2alt}}}(pressure)
 & 
Determine altitude from site pressure
\\

{\hyperref[stubs/pvlib.pvl_getaoi:pvlib.pvl_getaoi]{\code{pvlib.pvl\_getaoi}}}(SurfTilt, SurfAz, SunZen, SunAz)
 & 
Determine angle of incidence from surface tilt/azimuth and apparent sun zenith/azimuth
\\

{\hyperref[stubs/pvlib.pvl_disc:pvlib.pvl_disc]{\code{pvlib.pvl\_disc}}}(GHI, SunZen, Time{[}, pressure{]})
 & 
Estimate Direct Normal Irradiance from Global Horizontal Irradiance using the DISC model
\\

{\hyperref[stubs/pvlib.pvl_ephemeris:pvlib.pvl_ephemeris]{\code{pvlib.pvl\_ephemeris}}}(Time, Location{[}, ...{]})
 & 
Calculates the position of the sun given time, location, and optionally pressure and temperature
\\

{\hyperref[stubs/pvlib.pvl_spa:pvlib.pvl_spa]{\code{pvlib.pvl\_spa}}}(Time, Location)
 & 
Calculate the solar position using the PySolar package
\\

{\hyperref[stubs/pvlib.pvl_extraradiation:pvlib.pvl_extraradiation]{\code{pvlib.pvl\_extraradiation}}}(doy)
 & 
Determine extraterrestrial radiation from day of year
\\

{\hyperref[stubs/pvlib.pvl_globalinplane:pvlib.pvl_globalinplane]{\code{pvlib.pvl\_globalinplane}}}(SurfTilt, SurfAz, ...)
 & 
Determine the three components on in-plane irradiance
\\

{\hyperref[stubs/pvlib.pvl_grounddiffuse:pvlib.pvl_grounddiffuse]{\code{pvlib.pvl\_grounddiffuse}}}(SurfTilt, GHI, Albedo)
 & 
Estimate diffuse irradiance from ground reflections given irradiance, albedo, and surface tilt
\\

{\hyperref[stubs/pvlib.pvl_makelocationstruct:pvlib.pvl_makelocationstruct]{\code{pvlib.pvl\_makelocationstruct}}}(latitude, ...)
 & 
Create a struct to define a site location
\\

{\hyperref[stubs/pvlib.pvl_relativeairmass:pvlib.pvl_relativeairmass]{\code{pvlib.pvl\_relativeairmass}}}(z{[}, model{]})
 & 
Gives the relative (not pressure-corrected) airmass
\\

{\hyperref[stubs/pvlib.pvl_absoluteairmass:pvlib.pvl_absoluteairmass]{\code{pvlib.pvl\_absoluteairmass}}}(AMrelative, Pressure)
 & 
Determine absolute (pressure corrected) airmass from relative airmass and pressure
\\

{\hyperref[stubs/pvlib.pvl_clearsky_ineichen:pvlib.pvl_clearsky_ineichen]{\code{pvlib.pvl\_clearsky\_ineichen}}}(Time, Location)
 & 
Determine clear sky GHI, DNI, and DHI from Ineichen/Perez model
\\

{\hyperref[stubs/pvlib.pvl_clearsky_haurwitz:pvlib.pvl_clearsky_haurwitz]{\code{pvlib.pvl\_clearsky\_haurwitz}}}(ApparentZenith)
 & 
Determine clear sky GHI from Haurwitz model
\\
\hline\end{longtable}



\section{pvlib.pvl\_alt2pres}
\label{stubs/pvlib.pvl_alt2pres::doc}\label{stubs/pvlib.pvl_alt2pres:pvlib-pvl-alt2pres}\index{pvl\_alt2pres() (in module pvlib)}

\begin{fulllineitems}
\phantomsection\label{stubs/pvlib.pvl_alt2pres:pvlib.pvl_alt2pres}\pysiglinewithargsret{\code{pvlib.}\bfcode{pvl\_alt2pres}}{\emph{altitude}}{}
Determine site pressure from altitude
\begin{quote}\begin{description}
\item[{Parameters}] \leavevmode
\textbf{Altitude: scalar, vector or DataFrame} :
\begin{quote}

Altitude in meters above sea level
\end{quote}

\item[{Returns}] \leavevmode
\textbf{Pressure} : scalar, vector or DataFrame
\begin{quote}

Atomspheric pressure (Pascals)
\end{quote}

\end{description}\end{quote}


\strong{See also:}


{\hyperref[stubs/pvlib.pvl_alt2pres:pvlib.pvl_alt2pres]{\code{pvl\_alt2pres}}}, {\hyperref[stubs/pvlib.pvl_makelocationstruct:pvlib.pvl_makelocationstruct]{\code{pvl\_makelocationstruct}}}


\paragraph{Notes}

The following assumptions are made
\begin{quote}

\begin{tabulary}{\linewidth}{|L|L|}
\hline
\textsf{\relax 
Parameter
} & \textsf{\relax 
Value
}\\
\hline
Base pressure
 & 
101325 Pa
\\

Temperature at zero altitude
 & 
288.15 K
\\

Gravitational acceleration
 & 
9.80665 m/s\textasciicircum{}2
\\

Lapse rate
 & 
-6.5E-3 K/m
\\

Gas constant for air
 & 
287.053 J/(kgK)
\\

Relative Humidity
 & 
0\%
\\
\hline\end{tabulary}

\end{quote}
\paragraph{References}

``A Quick Derivation relating altitude to air pressure'' from Portland
State Aerospace Society, Version 1.03, 12/22/2004.

\end{fulllineitems}



\section{pvlib.pvl\_pres2alt}
\label{stubs/pvlib.pvl_pres2alt:pvlib-pvl-pres2alt}\label{stubs/pvlib.pvl_pres2alt::doc}\index{pvl\_pres2alt() (in module pvlib)}

\begin{fulllineitems}
\phantomsection\label{stubs/pvlib.pvl_pres2alt:pvlib.pvl_pres2alt}\pysiglinewithargsret{\code{pvlib.}\bfcode{pvl\_pres2alt}}{\emph{pressure}}{}
Determine altitude from site pressure
\begin{quote}\begin{description}
\item[{Parameters}] \leavevmode
\textbf{Pressure} : scalar, vector or DataFrame
\begin{quote}

Atomspheric pressure (Pascals)
\end{quote}

\item[{Returns}] \leavevmode
\textbf{altitude: scalar, vector or DataFrame} :
\begin{quote}

Altitude in meters above sea level
\end{quote}

\end{description}\end{quote}


\strong{See also:}


{\hyperref[stubs/pvlib.pvl_alt2pres:pvlib.pvl_alt2pres]{\code{pvl\_alt2pres}}}, {\hyperref[stubs/pvlib.pvl_makelocationstruct:pvlib.pvl_makelocationstruct]{\code{pvl\_makelocationstruct}}}


\paragraph{Notes}

The following assumptions are made
\begin{quote}

\begin{tabulary}{\linewidth}{|L|L|}
\hline
\textsf{\relax 
Parameter
} & \textsf{\relax 
Value
}\\
\hline
Base pressure
 & 
101325 Pa
\\

Temperature at zero altitude
 & 
288.15 K
\\

Gravitational acceleration
 & 
9.80665 m/s\textasciicircum{}2
\\

Lapse rate
 & 
-6.5E-3 K/m
\\

Gas constant for air
 & 
287.053 J/(kgK)
\\

Relative Humidity
 & 
0\%
\\
\hline\end{tabulary}

\end{quote}
\paragraph{References}

``A Quick Derivation relating altitude to air pressure'' from Portland
State Aerospace Society, Version 1.03, 12/22/2004.

\end{fulllineitems}



\section{pvlib.pvl\_getaoi}
\label{stubs/pvlib.pvl_getaoi:pvlib-pvl-getaoi}\label{stubs/pvlib.pvl_getaoi::doc}\index{pvl\_getaoi() (in module pvlib)}

\begin{fulllineitems}
\phantomsection\label{stubs/pvlib.pvl_getaoi:pvlib.pvl_getaoi}\pysiglinewithargsret{\code{pvlib.}\bfcode{pvl\_getaoi}}{\emph{SurfTilt}, \emph{SurfAz}, \emph{SunZen}, \emph{SunAz}}{}
Determine angle of incidence from surface tilt/azimuth and apparent sun zenith/azimuth

The surface is defined by its tilt angle from horizontal and its azimuth pointing angle. 
The sun position is defined by the apparent (refraction corrected)sun zenith angle and the sun 
azimuth angle.
\begin{quote}\begin{description}
\item[{Parameters}] \leavevmode
\textbf{SurfTilt} :  scalar or DataFrame of surface tilt angles in decimal degrees
\begin{quote}

If SurfTilt is a DataFrame it must be of the same size as all other DataFrame
inputs. SurfTilt must be \textgreater{}=0 and \textless{}=180. The tilt angle is defined as
degrees from horizontal (e.g. surface facing up = 0, surface facing
horizon = 90)
\end{quote}

\textbf{SurfAz} :  scalar or DataFrame of the surface azimuth angles in decimal degrees
\begin{quote}

If SurfAz is a DataFrame it must be of the same size as all other DataFrame
inputs. SurfAz must be \textgreater{}=0 and \textless{}=360. The Azimuth convention is defined
as degrees east of north (e.g. North = 0, East = 90, West = 270).
\end{quote}

\textbf{SunZen} : scalar or DataFrame of apparent (refraction-corrected) zenith angles in decimal degrees.
\begin{quote}

If SunZen is a DataFrame it must be of the same size as all other DataFrame 
inputs. SunZen must be \textgreater{}=0 and \textless{}=180.
\end{quote}

\textbf{SunAz} : scalar or DataFrame of sun azimuth angles in decimal degrees
\begin{quote}

If SunAz is a DataFrame it must be of the same size as all other DataFrame
inputs. SunAz must be \textgreater{}=0 and \textless{}=360. The Azimuth convention is defined
as degrees east of north (e.g. North = 0, East = 90, West = 270).
\end{quote}

\item[{Returns}] \leavevmode
\textbf{AOI} : DataFrame
\begin{quote}

The angle, in decimal degrees, between the surface normal DataFrame and the sun beam DataFrame.
\end{quote}

\end{description}\end{quote}


\strong{See also:}


\code{PVL\_EPHEMERIS}


\paragraph{References}

D.L. King, J.A. Kratochvil, W.E. Boyson. ``Spectral and
Angle-of-Incidence Effects on Photovoltaic Modules and Solar Irradiance
Sensors''. 26th IEEE Photovoltaic Specialists Conference. Sept. 1997.

\end{fulllineitems}



\section{pvlib.pvl\_disc}
\label{stubs/pvlib.pvl_disc::doc}\label{stubs/pvlib.pvl_disc:pvlib-pvl-disc}\index{pvl\_disc() (in module pvlib)}

\begin{fulllineitems}
\phantomsection\label{stubs/pvlib.pvl_disc:pvlib.pvl_disc}\pysiglinewithargsret{\code{pvlib.}\bfcode{pvl\_disc}}{\emph{GHI}, \emph{SunZen}, \emph{Time}, \emph{pressure=101325}}{}
Estimate Direct Normal Irradiance from Global Horizontal Irradiance using the DISC model

The DISC algorithm converts global horizontal irradiance to direct
normal irradiance through empirical relationships between the global
and direct clearness indices.
\begin{quote}\begin{description}
\item[{Parameters}] \leavevmode
\textbf{GHI} : float or DataFrame
\begin{quote}

global horizontal irradiance in W/m\textasciicircum{}2. GHI must be \textgreater{}=0.
\end{quote}

\textbf{Z} : float or DataFrame
\begin{quote}

True (not refraction - corrected) zenith angles in decimal degrees. 
Z must be \textgreater{}=0 and \textless{}=180.
\end{quote}

\textbf{doy} : float or DataFrame
\begin{quote}

the day of the year. doy must be \textgreater{}= 1 and \textless{} 367.
\end{quote}

\item[{Returns}] \leavevmode
\textbf{DNI} : float or DataFrame
\begin{quote}

The modeled direct normal irradiance in W/m\textasciicircum{}2 provided by the
Direct Insolation Simulation Code (DISC) model.
\end{quote}

\textbf{Kt} : float or DataFrame
\begin{quote}

Ratio of global to extraterrestrial irradiance on a horizontal plane.
\end{quote}

\item[{Other Parameters}] \leavevmode
\textbf{pressure} : float or DataFrame (optional, Default=101325)
\begin{quote}

site pressure in Pascal. Pressure may be measured
or an average pressure may be calculated from site altitude. If
pressure is omitted, standard pressure (101325 Pa) will be used, this
is acceptable if the site is near sea level. If the site is not near
sea:level, inclusion of a measured or average pressure is highly
recommended.
\end{quote}

\end{description}\end{quote}


\strong{See also:}


{\hyperref[stubs/pvlib.pvl_ephemeris:pvlib.pvl_ephemeris]{\code{pvl\_ephemeris}}}, {\hyperref[stubs/pvlib.pvl_alt2pres:pvlib.pvl_alt2pres]{\code{pvl\_alt2pres}}}, \code{pvl\_dirint}


\paragraph{References}

{[}1{]} Maxwell, E. L., ``A Quasi-Physical Model for Converting Hourly 
Global Horizontal to Direct Normal Insolation'', Technical 
Report No. SERI/TR-215-3087, Golden, CO: Solar Energy Research 
Institute, 1987.

{[}2{]} J.W. ``Fourier series representation of the position of the sun''. 
Found at:
\href{http://www.mail-archive.com/sundial@uni-koeln.de/msg01050.html}{http://www.mail-archive.com/sundial@uni-koeln.de/msg01050.html} on
January 12, 2012

\end{fulllineitems}



\section{pvlib.pvl\_ephemeris}
\label{stubs/pvlib.pvl_ephemeris:pvlib-pvl-ephemeris}\label{stubs/pvlib.pvl_ephemeris::doc}\index{pvl\_ephemeris() (in module pvlib)}

\begin{fulllineitems}
\phantomsection\label{stubs/pvlib.pvl_ephemeris:pvlib.pvl_ephemeris}\pysiglinewithargsret{\code{pvlib.}\bfcode{pvl\_ephemeris}}{\emph{Time}, \emph{Location}, \emph{pressure=101325}, \emph{temperature=12}}{}
Calculates the position of the sun given time, location, and optionally pressure and temperature
\begin{quote}\begin{description}
\item[{Parameters}] \leavevmode
\textbf{Time} :  \href{http://pandas.pydata.org/pandas-docs/version/0.13.1/generated/pandas.Index.html}{pandas.Index}

\textbf{Location: struct} :
\begin{quote}
\begin{description}
\item[{\emph{Location.latitude} - vector or scalar latitude in decimal degrees (positive is}] \leavevmode
northern hemisphere)

\item[{\emph{Location.longitude} - vector or scalar longitude in decimal degrees (positive is }] \leavevmode
east of prime meridian)

\item[{\emph{Location.altitude} - an optional component of the Location struct, not}] \leavevmode
used in the ephemeris code directly, but it may be used to calculate
standard site pressure (see pvl\_alt2pres function)

\end{description}

\emph{location.TZ}     - Time Zone offset from UTC
\end{quote}

\item[{Returns}] \leavevmode
\textbf{SunAz} : float of DataFrame
\begin{quote}

Azimuth of the sun in decimal degrees from North. 0 = North to 270 = West
\end{quote}

\textbf{SunEl} : float of DataFrame
\begin{quote}

Actual elevation (not accounting for refraction)of the sun 
in decimal degrees, 0 = on horizon. The complement of the True Zenith
Angle.
\end{quote}

\textbf{ApparentSunEl} : float or DataFrame
\begin{quote}

Apparent sun elevation accounting for atmospheric 
refraction. This is the complement of the Apparent Zenith Angle.
\end{quote}

\textbf{SolarTime} : fload or DataFrame
\begin{quote}

Solar time in decimal hours (solar noon is 12.00).
\end{quote}

\item[{Other Parameters}] \leavevmode
\textbf{pressure} : float or DataFrame
\begin{quote}

Ambient pressure (Pascals)
\end{quote}

\textbf{tempreature: float or DataFrame} :
\begin{quote}

Ambient temperature (C)
\end{quote}

\end{description}\end{quote}


\strong{See also:}


{\hyperref[stubs/pvlib.pvl_makelocationstruct:pvlib.pvl_makelocationstruct]{\code{pvl\_makelocationstruct}}}, {\hyperref[stubs/pvlib.pvl_alt2pres:pvlib.pvl_alt2pres]{\code{pvl\_alt2pres}}}, {\hyperref[stubs/pvlib.pvl_getaoi:pvlib.pvl_getaoi]{\code{pvl\_getaoi}}}, {\hyperref[stubs/pvlib.pvl_spa:pvlib.pvl_spa]{\code{pvl\_spa}}}


\paragraph{References}

Grover Hughes' class and related class materials on Engineering 
Astronomy at Sandia National Laboratories, 1985.

\end{fulllineitems}



\section{pvlib.pvl\_spa}
\label{stubs/pvlib.pvl_spa::doc}\label{stubs/pvlib.pvl_spa:pvlib-pvl-spa}\index{pvl\_spa() (in module pvlib)}

\begin{fulllineitems}
\phantomsection\label{stubs/pvlib.pvl_spa:pvlib.pvl_spa}\pysiglinewithargsret{\code{pvlib.}\bfcode{pvl\_spa}}{\emph{Time}, \emph{Location}}{}
Calculate the solar position using the PySolar package

The PySolar Package is developed by Brandon Stafford, and is
found here: \href{https://github.com/pingswept/pysolar/tree/master}{https://github.com/pingswept/pysolar/tree/master}

This function will map the standard time and location structures
onto the Pysolar function
\begin{quote}\begin{description}
\item[{Parameters}] \leavevmode
\textbf{Time: Dataframe.index} :
\begin{quote}

A pandas datatime object
\end{quote}

\textbf{Location: struct} :
\begin{quote}

Standard location structure, containting:
\begin{description}
\item[{\emph{Location.latitude} - vector or scalar latitude in decimal degrees (positive is}] \leavevmode
northern hemisphere)

\item[{\emph{Location.longitude} - vector or scalar longitude in decimal degrees (positive is }] \leavevmode
east of prime meridian)

\item[{\emph{Location.altitude} - an optional component of the Location struct, not}] \leavevmode
used in the ephemeris code directly, but it may be used to calculate
standard site pressure (see pvl\_alt2pres function)

\end{description}
\end{quote}

\item[{Returns}] \leavevmode
\textbf{SunAz} : DataFrame
\begin{quote}

Azimuth of the sun in decimal degrees from North. 0 = North to 270 = West
\end{quote}

\textbf{SunEl} : DataFrame
\begin{quote}

Actual elevation (not accounting for refraction)of the sun 
in decimal degrees, 0 = on horizon. The complement of the True Zenith
Angle.
\end{quote}

\end{description}\end{quote}
\paragraph{References}

PySolar Documentation: \href{https://github.com/pingswept/pysolar/tree/master}{https://github.com/pingswept/pysolar/tree/master}

\end{fulllineitems}



\section{pvlib.pvl\_extraradiation}
\label{stubs/pvlib.pvl_extraradiation:pvlib-pvl-extraradiation}\label{stubs/pvlib.pvl_extraradiation::doc}\index{pvl\_extraradiation() (in module pvlib)}

\begin{fulllineitems}
\phantomsection\label{stubs/pvlib.pvl_extraradiation:pvlib.pvl_extraradiation}\pysiglinewithargsret{\code{pvlib.}\bfcode{pvl\_extraradiation}}{\emph{doy}}{}
Determine extraterrestrial radiation from day of year
\begin{quote}\begin{description}
\item[{Parameters}] \leavevmode
\textbf{doy} : int or pandas.index.dayofyear
\begin{quote}

Day of the year
\end{quote}

\item[{Returns}] \leavevmode
\textbf{Ea} : float or DataFrame
\begin{quote}

the extraterrestrial radiation present in watts per square meter
on a surface which is normal to the sun. Ea is of the same size as the
input doy.
\end{quote}

\end{description}\end{quote}


\strong{See also:}


{\hyperref[stubs/pvlib.pvl_disc:pvlib.pvl_disc]{\code{pvl\_disc}}}


\paragraph{References}

\textless{}\href{http://solardat.uoregon.edu/SolarRadiationBasics.html}{http://solardat.uoregon.edu/SolarRadiationBasics.html}\textgreater{}, Eqs. SR1 and SR2

SR1       Partridge, G. W. and Platt, C. M. R. 1976. Radiative Processes in Meteorology and Climatology.

SR2       Duffie, J. A. and Beckman, W. A. 1991. Solar Engineering of Thermal Processes, 2nd edn. J. Wiley and Sons, New York.

\end{fulllineitems}



\section{pvlib.pvl\_globalinplane}
\label{stubs/pvlib.pvl_globalinplane:pvlib-pvl-globalinplane}\label{stubs/pvlib.pvl_globalinplane::doc}\index{pvl\_globalinplane() (in module pvlib)}

\begin{fulllineitems}
\phantomsection\label{stubs/pvlib.pvl_globalinplane:pvlib.pvl_globalinplane}\pysiglinewithargsret{\code{pvlib.}\bfcode{pvl\_globalinplane}}{\emph{SurfTilt}, \emph{SurfAz}, \emph{AOI}, \emph{DNI}, \emph{In\_Plane\_SkyDiffuse}, \emph{GR}}{}
Determine the three components on in-plane irradiance

Combines in-plane irradaince compoents from the chosen diffuse translation, ground 
reflection and beam irradiance algorithms into the total in-plane irradiance.
\begin{quote}\begin{description}
\item[{Parameters}] \leavevmode
\textbf{SurfTilt} : float or DataFrame
\begin{quote}

surface tilt angles in decimal degrees.
SurfTilt must be \textgreater{}=0 and \textless{}=180. The tilt angle is defined as
degrees from horizontal (e.g. surface facing up = 0, surface facing
horizon = 90)
\end{quote}

\textbf{SurfAz} : float or DataFrame
\begin{quote}

Surface azimuth angles in decimal degrees.
SurfAz must be \textgreater{}=0 and \textless{}=360. The Azimuth convention is defined
as degrees east of north (e.g. North = 0, south=180, East = 90, West = 270).
\end{quote}

\textbf{AOI} : float or DataFrame
\begin{quote}

Angle of incedence of solar rays with respect
to the module surface, from \code{pvl\_getaoi}. AOI must be \textgreater{}=0 and \textless{}=180.
\end{quote}

\textbf{DNI} : float or DataFrame
\begin{quote}

Direct normal irradiance (W/m\textasciicircum{}2), as measured 
from a TMY file or calculated with a clearsky model.
\end{quote}

\textbf{In\_Plane\_SkyDiffuse} :  float or DataFrame
\begin{quote}

Diffuse irradiance (W/m\textasciicircum{}2) in the plane of the modules, as
calculated by a diffuse irradiance translation function
\end{quote}

\textbf{GR} : float or DataFrame
\begin{quote}

a scalar or DataFrame of ground reflected irradiance (W/m\textasciicircum{}2), as calculated
by a albedo model (eg. \code{pvl\_grounddiffuse})
\end{quote}

\item[{Returns}] \leavevmode
\textbf{E} : float or DataFrame
\begin{quote}

Total in-plane irradiance (W/m\textasciicircum{}2)
\end{quote}

\textbf{Eb} : float or DataFrame
\begin{quote}

Total in-plane beam irradiance (W/m\textasciicircum{}2)
\end{quote}

\textbf{Ediff} : float or DataFrame
\begin{quote}

Total in-plane diffuse irradiance (W/m\textasciicircum{}2)
\end{quote}

\end{description}\end{quote}


\strong{See also:}


{\hyperref[stubs/pvlib.pvl_grounddiffuse:pvlib.pvl_grounddiffuse]{\code{pvl\_grounddiffuse}}}, {\hyperref[stubs/pvlib.pvl_getaoi:pvlib.pvl_getaoi]{\code{pvl\_getaoi}}}, {\hyperref[stubs/pvlib.pvl_perez:pvlib.pvl_perez]{\code{pvl\_perez}}}, {\hyperref[stubs/pvlib.pvl_reindl1990:pvlib.pvl_reindl1990]{\code{pvl\_reindl1990}}}, {\hyperref[stubs/pvlib.pvl_klucher1979:pvlib.pvl_klucher1979]{\code{pvl\_klucher1979}}}, {\hyperref[stubs/pvlib.pvl_haydavies1980:pvlib.pvl_haydavies1980]{\code{pvl\_haydavies1980}}}, {\hyperref[stubs/pvlib.pvl_isotropicsky:pvlib.pvl_isotropicsky]{\code{pvl\_isotropicsky}}}, {\hyperref[stubs/pvlib.pvl_kingdiffuse:pvlib.pvl_kingdiffuse]{\code{pvl\_kingdiffuse}}}



\end{fulllineitems}



\section{pvlib.pvl\_grounddiffuse}
\label{stubs/pvlib.pvl_grounddiffuse::doc}\label{stubs/pvlib.pvl_grounddiffuse:pvlib-pvl-grounddiffuse}\index{pvl\_grounddiffuse() (in module pvlib)}

\begin{fulllineitems}
\phantomsection\label{stubs/pvlib.pvl_grounddiffuse:pvlib.pvl_grounddiffuse}\pysiglinewithargsret{\code{pvlib.}\bfcode{pvl\_grounddiffuse}}{\emph{SurfTilt}, \emph{GHI}, \emph{Albedo}}{}
Estimate diffuse irradiance from ground reflections given irradiance, albedo, and surface tilt

Function to determine the portion of irradiance on a tilted surface due
to ground reflections. Any of the inputs may be DataFrames or scalars.
\begin{quote}\begin{description}
\item[{Parameters}] \leavevmode
\textbf{SurfTilt} : float or DataFrame
\begin{quote}
\begin{quote}

Surface tilt angles in decimal degrees.
\end{quote}

SurfTilt must be \textgreater{}=0 and \textless{}=180. The tilt angle is defined as
degrees from horizontal (e.g. surface facing up = 0, surface facing
horizon = 90).
\end{quote}

\textbf{GHI} : float or DataFrame
\begin{quote}

Global horizontal irradiance in W/m\textasciicircum{}2.  
GHI must be \textgreater{}=0.
\end{quote}

\textbf{Albedo} : float or DataFrame
\begin{quote}

Ground reflectance, typically 0.1-0.4 for
surfaces on Earth (land), may increase over snow, ice, etc. May also 
be known as the reflection coefficient. Must be \textgreater{}=0 and \textless{}=1.
\end{quote}

\item[{Returns}] \leavevmode
\textbf{GR} : float or DataFrame
\begin{quote}

Ground reflected irradiances in W/m\textasciicircum{}2.
\end{quote}

\end{description}\end{quote}


\strong{See also:}


{\hyperref[stubs/pvlib.pvl_disc:pvlib.pvl_disc]{\code{pvl\_disc}}}, {\hyperref[stubs/pvlib.pvl_perez:pvlib.pvl_perez]{\code{pvl\_perez}}}, {\hyperref[stubs/pvlib.pvl_reindl1990:pvlib.pvl_reindl1990]{\code{pvl\_reindl1990}}}, {\hyperref[stubs/pvlib.pvl_klucher1979:pvlib.pvl_klucher1979]{\code{pvl\_klucher1979}}}, {\hyperref[stubs/pvlib.pvl_haydavies1980:pvlib.pvl_haydavies1980]{\code{pvl\_haydavies1980}}}, {\hyperref[stubs/pvlib.pvl_isotropicsky:pvlib.pvl_isotropicsky]{\code{pvl\_isotropicsky}}}, {\hyperref[stubs/pvlib.pvl_kingdiffuse:pvlib.pvl_kingdiffuse]{\code{pvl\_kingdiffuse}}}


\paragraph{References}

{[}1{]} Loutzenhiser P.G. et. al. ``Empirical validation of models to compute
solar irradiance on inclined surfaces for building energy simulation''
2007, Solar Energy vol. 81. pp. 254-267

\end{fulllineitems}



\section{pvlib.pvl\_makelocationstruct}
\label{stubs/pvlib.pvl_makelocationstruct:pvlib-pvl-makelocationstruct}\label{stubs/pvlib.pvl_makelocationstruct::doc}\index{pvl\_makelocationstruct() (in module pvlib)}

\begin{fulllineitems}
\phantomsection\label{stubs/pvlib.pvl_makelocationstruct:pvlib.pvl_makelocationstruct}\pysiglinewithargsret{\code{pvlib.}\bfcode{pvl\_makelocationstruct}}{\emph{latitude}, \emph{longitude}, \emph{TZ}, \emph{altitude=100}}{}
Create a struct to define a site location
\begin{quote}\begin{description}
\item[{Parameters}] \leavevmode
\textbf{Latitude} : float
\begin{quote}

Positive north of equator, decimal notation
\end{quote}

\textbf{Longitude} : float
\begin{quote}

Positive east of prime meridian, decimal notation
\end{quote}

\textbf{TZ} : int
\begin{quote}

Timezone in GMT offset
\end{quote}

\item[{Returns}] \leavevmode
\textbf{Location} : struct
\begin{quote}

\emph{Location.latitude}

\emph{Location.longitude}

\emph{Location.TZ}

\emph{Location.altitude}
\end{quote}

\item[{Other Parameters}] \leavevmode
\textbf{altitude} : float (optional, default=100)
\begin{quote}

Altitude from sea level. Set to 100m if none input
\end{quote}

\end{description}\end{quote}


\strong{See also:}


{\hyperref[stubs/pvlib.pvl_ephemeris:pvlib.pvl_ephemeris]{\code{pvl\_ephemeris}}}, {\hyperref[stubs/pvlib.pvl_alt2pres:pvlib.pvl_alt2pres]{\code{pvl\_alt2pres}}}, {\hyperref[stubs/pvlib.pvl_pres2alt:pvlib.pvl_pres2alt]{\code{pvl\_pres2alt}}}



\end{fulllineitems}



\section{pvlib.pvl\_relativeairmass}
\label{stubs/pvlib.pvl_relativeairmass:pvlib-pvl-relativeairmass}\label{stubs/pvlib.pvl_relativeairmass::doc}\index{pvl\_relativeairmass() (in module pvlib)}

\begin{fulllineitems}
\phantomsection\label{stubs/pvlib.pvl_relativeairmass:pvlib.pvl_relativeairmass}\pysiglinewithargsret{\code{pvlib.}\bfcode{pvl\_relativeairmass}}{\emph{z}, \emph{model='kastenyoung1989'}}{}
Gives the relative (not pressure-corrected) airmass
\begin{quote}

Gives the airmass at sea-level when given a sun zenith angle, z (in 
degrees). 
The ``model'' variable allows selection of different airmass models
(described below). ``model'' must be a valid string. If ``model'' is not 
included or is not valid, the default model is `kastenyoung1989'.
\end{quote}
\begin{quote}\begin{description}
\item[{Parameters}] \leavevmode
\textbf{z} : float or DataFrame
\begin{quote}

Zenith angle of the sun.  Note that some models use the apparent (refraction corrected)
zenith angle, and some models use the true (not refraction-corrected)
zenith angle. See model descriptions to determine which type of zenith
angle is required.
\end{quote}

\textbf{model} : String
\begin{quote}

Avaiable models include the following:
\begin{itemize}
\item {} 
`simple' - secant(apparent zenith angle) - Note that this gives -inf at zenith=90

\item {} 
`kasten1966' - See reference {[}1{]} - requires apparent sun zenith

\item {} 
`youngirvine1967' - See reference {[}2{]} - requires true sun zenith

\item {} 
`kastenyoung1989' - See reference {[}3{]} - requires apparent sun zenith

\item {} 
`gueymard1993' - See reference {[}4{]} - requires apparent sun zenith

\item {} 
`young1994' - See reference {[}5{]} - requries true sun zenith

\item {} 
`pickering2002' - See reference {[}6{]} - requires apparent sun zenith

\end{itemize}
\end{quote}

\item[{Returns}] \leavevmode
\textbf{AM} : float or DataFrame
\begin{quote}

Relative airmass at sea level.  Will return NaN values for all zenith 
angles greater than 90 degrees.
\end{quote}

\end{description}\end{quote}


\strong{See also:}


{\hyperref[stubs/pvlib.pvl_absoluteairmass:pvlib.pvl_absoluteairmass]{\code{pvl\_absoluteairmass}}}, {\hyperref[stubs/pvlib.pvl_ephemeris:pvlib.pvl_ephemeris]{\code{pvl\_ephemeris}}}


\paragraph{References}

{[}1{]} Fritz Kasten. ``A New Table and Approximation Formula for the
Relative Optical Air Mass''. Technical Report 136, Hanover, N.H.: U.S.
Army Material Command, CRREL.

{[}2{]} A. T. Young and W. M. Irvine, ``Multicolor Photoelectric Photometry
of the Brighter Planets,'' The Astronomical Journal, vol. 72, 
pp. 945-950, 1967.

{[}3{]} Fritz Kasten and Andrew Young. ``Revised optical air mass tables and
approximation formula''. Applied Optics 28:4735-4738

{[}4{]} C. Gueymard, ``Critical analysis and performance assessment of 
clear sky solar irradiance models using theoretical and measured data,''
Solar Energy, vol. 51, pp. 121-138, 1993.

{[}5{]} A. T. Young, ``AIR-MASS AND REFRACTION,'' Applied Optics, vol. 33, 
pp. 1108-1110, Feb 1994.

{[}6{]} Keith A. Pickering. ``The Ancient Star Catalog''. DIO 12:1, 20,

\end{fulllineitems}



\section{pvlib.pvl\_absoluteairmass}
\label{stubs/pvlib.pvl_absoluteairmass::doc}\label{stubs/pvlib.pvl_absoluteairmass:pvlib-pvl-absoluteairmass}\index{pvl\_absoluteairmass() (in module pvlib)}

\begin{fulllineitems}
\phantomsection\label{stubs/pvlib.pvl_absoluteairmass:pvlib.pvl_absoluteairmass}\pysiglinewithargsret{\code{pvlib.}\bfcode{pvl\_absoluteairmass}}{\emph{AMrelative}, \emph{Pressure}}{}
Determine absolute (pressure corrected) airmass from relative airmass and pressure

Gives the airmass for locations not at sea-level (i.e. not at standard
pressure). The input argument ``AMrelative'' is the relative airmass. The
input argument ``pressure'' is the pressure (in Pascals) at the location
of interest and must be greater than 0. The calculation for
absolute airmass is:
absolute airmass = (relative airmass)*pressure/101325
\begin{quote}\begin{description}
\item[{Parameters}] \leavevmode
\textbf{AMrelative} : float or DataFrame
\begin{quote}

The airmass at sea-level which can be calculated using the 
PV\_LIB function pvl\_relativeairmass.
\end{quote}

\textbf{pressure} : float or DataFrame
\begin{quote}

a scalar or vector of values providing the site pressure in
Pascal. If pressure is a vector it must be of the same size as all
other vector inputs. pressure must be \textgreater{}=0. Pressure may be measured
or an average pressure may be calculated from site altitude.
\end{quote}

\item[{Returns}] \leavevmode
\textbf{AMa} : float or DataFrame
\begin{quote}

Absolute (pressure corrected) airmass
\end{quote}

\end{description}\end{quote}


\strong{See also:}


{\hyperref[stubs/pvlib.pvl_relativeairmass:pvlib.pvl_relativeairmass]{\code{pvl\_relativeairmass}}}


\paragraph{References}

{[}1{]} C. Gueymard, ``Critical analysis and performance assessment of 
clear sky solar irradiance models using theoretical and measured data,''
Solar Energy, vol. 51, pp. 121-138, 1993.

\end{fulllineitems}



\section{pvlib.pvl\_clearsky\_ineichen}
\label{stubs/pvlib.pvl_clearsky_ineichen::doc}\label{stubs/pvlib.pvl_clearsky_ineichen:pvlib-pvl-clearsky-ineichen}\index{pvl\_clearsky\_ineichen() (in module pvlib)}

\begin{fulllineitems}
\phantomsection\label{stubs/pvlib.pvl_clearsky_ineichen:pvlib.pvl_clearsky_ineichen}\pysiglinewithargsret{\code{pvlib.}\bfcode{pvl\_clearsky\_ineichen}}{\emph{Time}, \emph{Location}, \emph{LinkeTurbidity=-999}}{}
Determine clear sky GHI, DNI, and DHI from Ineichen/Perez model

Implements the Ineichen and Perez clear sky model for global horizontal
irradiance (GHI), direct normal irradiance (DNI), and calculates
the clear-sky diffuse horizontal (DHI) component as the difference
between GHI and DNI*cos(zenith) as presented in {[}1, 2{]}. A report on clear
sky models found the Ineichen/Perez model to have excellent performance
with a minimal input data set {[}3{]}. Default values for Linke turbidity
provided by SoDa {[}4, 5{]}.
\begin{quote}\begin{description}
\item[{Parameters}] \leavevmode
\textbf{Time} : Dataframe.index
\begin{quote}

A timezone aware pandas dataframe index.
\end{quote}

\textbf{Location} : struct
\begin{quote}
\begin{description}
\item[{\emph{Location.latitude} - vector or scalar latitude in decimal degrees (positive is}] \leavevmode
northern hemisphere)

\item[{\emph{Location.longitude} - vector or scalar longitude in decimal degrees (positive is }] \leavevmode
east of prime meridian)

\item[{\emph{Location.altitude} - an optional component of the Location struct, not}] \leavevmode
used in the ephemeris code directly, but it may be used to calculate
standard site pressure (see pvl\_alt2pres function)

\end{description}

\emph{location.TZ}     - Time Zone offset from UTC
\end{quote}

\item[{Returns}] \leavevmode
\textbf{ClearSkyGHI} : Dataframe
\begin{quote}
\begin{description}
\item[{the modeled global horizonal irradiance in W/m\textasciicircum{}2 provided}] \leavevmode
by the Ineichen clear-sky model.

\end{description}
\end{quote}

\textbf{ClearSkyDNI} : Dataframe
\begin{quote}

the modeled direct normal irradiance in W/m\textasciicircum{}2 provided
by the Ineichen clear-sky model.
\end{quote}

\textbf{ClearSkyDHI} : Dataframe
\begin{quote}

the calculated diffuse horizonal irradiance in W/m\textasciicircum{}2 
provided by the Ineichen clear-sky model.
\end{quote}

\item[{Other Parameters}] \leavevmode
\textbf{LinkeTurbidityInput} : Optional, float or DataFrame
\begin{quote}

An optional input to provide your own Linke
turbidity. If this input is omitted, the default Linke turbidity
maps will be used. LinkeTurbidityInput may be a float or 
dataframe of Linke turbidities. If dataframe is provided, the same
turbidity will be used for all time/location sets. If a dataframe is
provided, it must be of the same size as any time/location dataframes
and each element of the dataframe corresponds to any time and location
elements.
\end{quote}

\end{description}\end{quote}


\strong{See also:}


\code{pvl\_maketimestruct}, {\hyperref[stubs/pvlib.pvl_makelocationstruct:pvlib.pvl_makelocationstruct]{\code{pvl\_makelocationstruct}}}, {\hyperref[stubs/pvlib.pvl_ephemeris:pvlib.pvl_ephemeris]{\code{pvl\_ephemeris}}}, \code{pvl\_haurwitz}


\paragraph{Notes}

This implementation of the Ineichen model requires a number of other
PV\_LIB functions including pvl\_ephemeris, pvl\_date2doy,
pvl\_extraradiation, pvl\_absoluteairmass, pvl\_relativeairmass, and
pvl\_alt2pres. It also requires the file ``LinkeTurbidities.mat'' to be
in the working directory. If you are using pvl\_ineichen
in a loop, it may be faster to load LinkeTurbidities.mat outside of
the loop and feed it into pvl\_ineichen as a variable, rather than
having pvl\_ineichen open the file each time it is called (or utilize
column vectors of time/location instead of a loop).

Initial implementation of this algorithm by Matthew Reno.

\end{fulllineitems}



\section{pvlib.pvl\_clearsky\_haurwitz}
\label{stubs/pvlib.pvl_clearsky_haurwitz:pvlib-pvl-clearsky-haurwitz}\label{stubs/pvlib.pvl_clearsky_haurwitz::doc}\index{pvl\_clearsky\_haurwitz() (in module pvlib)}

\begin{fulllineitems}
\phantomsection\label{stubs/pvlib.pvl_clearsky_haurwitz:pvlib.pvl_clearsky_haurwitz}\pysiglinewithargsret{\code{pvlib.}\bfcode{pvl\_clearsky\_haurwitz}}{\emph{ApparentZenith}}{}
Determine clear sky GHI from Haurwitz model

Implements the Haurwitz clear sky model for global horizontal
irradiance (GHI) as presented in {[}1, 2{]}. A report on clear
sky models found the Haurwitz model to have the best performance of
models which require only zenith angle {[}3{]}.
\begin{quote}\begin{description}
\item[{Parameters}] \leavevmode
\textbf{ApparentZenith} : DataFrame
\begin{quote}
\begin{quote}

The apparent (refraction corrected) sun zenith angle
\end{quote}

in degrees.
\end{quote}

\item[{Returns}] \leavevmode
\textbf{ClearSkyGHI} : DataFrame
\begin{quote}
\begin{description}
\item[{the modeled global horizonal irradiance in W/m\textasciicircum{}2 provided}] \leavevmode
by the Haurwitz clear-sky model.
\begin{quote}

Initial implementation of this algorithm by Matthew Reno.
\end{quote}

\end{description}
\end{quote}

\end{description}\end{quote}


\strong{See also:}


\code{pvl\_maketimestruct}, {\hyperref[stubs/pvlib.pvl_makelocationstruct:pvlib.pvl_makelocationstruct]{\code{pvl\_makelocationstruct}}}, {\hyperref[stubs/pvlib.pvl_ephemeris:pvlib.pvl_ephemeris]{\code{pvl\_ephemeris}}}, {\hyperref[stubs/pvlib.pvl_spa:pvlib.pvl_spa]{\code{pvl\_spa}}}, \code{pvl\_ineichen}


\paragraph{References}
\begin{description}
\item[{{[}1{]} B. Haurwitz, ``Insolation in Relation to Cloudiness and Cloud }] \leavevmode
Density,'' Journal of Meteorology, vol. 2, pp. 154-166, 1945.

\item[{{[}2{]} B. Haurwitz, ``Insolation in Relation to Cloud Type,'' Journal of }] \leavevmode
Meteorology, vol. 3, pp. 123-124, 1946.

\item[{{[}3{]} M. Reno, C. Hansen, and J. Stein, ``Global Horizontal Irradiance Clear}] \leavevmode
Sky Models: Implementation and Analysis'', Sandia National
Laboratories, SAND2012-2389, 2012.

\end{description}

\end{fulllineitems}



\chapter{Irradiance Translation Functions}
\label{index:irradiance-translation-functions}
\begin{longtable}{lp{7cm}}
\hline
\endfirsthead

\multicolumn{2}{c}%
{{\textsf{\tablename\ \thetable{} -- continued from previous page}}} \\
\hline
\endhead

\hline \multicolumn{2}{|r|}{{\textsf{Continued on next page}}} \\ \hline
\endfoot

\endlastfoot


{\hyperref[stubs/pvlib.pvl_perez:pvlib.pvl_perez]{\code{pvlib.pvl\_perez}}}(SurfTilt, SurfAz, DHI, DNI, ...)
 & 
Determine diffuse irradiance from the sky on a tilted surface using one of the Perez models
\\

{\hyperref[stubs/pvlib.pvl_haydavies1980:pvlib.pvl_haydavies1980]{\code{pvlib.pvl\_haydavies1980}}}(SurfTilt, SurfAz, ...)
 & 
Determine diffuse irradiance from the sky on a tilted surface using Hay \& Davies' 1980 model
\\

{\hyperref[stubs/pvlib.pvl_isotropicsky:pvlib.pvl_isotropicsky]{\code{pvlib.pvl\_isotropicsky}}}(SurfTilt, DHI)
 & 
Determine diffuse irradiance from the sky on a tilted surface using isotropic sky model
\\

{\hyperref[stubs/pvlib.pvl_kingdiffuse:pvlib.pvl_kingdiffuse]{\code{pvlib.pvl\_kingdiffuse}}}(SurfTilt, DHI, GHI, SunZen)
 & 
Determine diffuse irradiance from the sky on a tilted surface using the King model
\\

{\hyperref[stubs/pvlib.pvl_klucher1979:pvlib.pvl_klucher1979]{\code{pvlib.pvl\_klucher1979}}}(SurfTilt, SurfAz, DHI, ...)
 & 
Determine diffuse irradiance from the sky on a tilted surface using Klucher's 1979 model
\\

{\hyperref[stubs/pvlib.pvl_reindl1990:pvlib.pvl_reindl1990]{\code{pvlib.pvl\_reindl1990}}}(SurfTilt, SurfAz, DHI, ...)
 & 
Determine diffuse irradiance from the sky on a tilted surface using Reindl's 1990 model
\\
\hline\end{longtable}



\section{pvlib.pvl\_perez}
\label{stubs/pvlib.pvl_perez:pvlib-pvl-perez}\label{stubs/pvlib.pvl_perez::doc}\index{pvl\_perez() (in module pvlib)}

\begin{fulllineitems}
\phantomsection\label{stubs/pvlib.pvl_perez:pvlib.pvl_perez}\pysiglinewithargsret{\code{pvlib.}\bfcode{pvl\_perez}}{\emph{SurfTilt}, \emph{SurfAz}, \emph{DHI}, \emph{DNI}, \emph{HExtra}, \emph{SunZen}, \emph{SunAz}, \emph{AM}, \emph{modelt='allsitescomposite1990'}}{}
Determine diffuse irradiance from the sky on a tilted surface using one of the Perez models

Perez models determine the diffuse irradiance from the sky (ground
reflected irradiance is not included in this algorithm) on a tilted
surface using the surface tilt angle, surface azimuth angle, diffuse
horizontal irradiance, direct normal irradiance, extraterrestrial
irradiance, sun zenith angle, sun azimuth angle, and relative (not
pressure-corrected) airmass. Optionally a selector may be used to use
any of Perez's model coefficient sets.
\begin{quote}\begin{description}
\item[{Parameters}] \leavevmode
\textbf{SurfTilt} : float or DataFrame
\begin{quote}

Surface tilt angles in decimal degrees.
SurfTilt must be \textgreater{}=0 and \textless{}=180. The tilt angle is defined as
degrees from horizontal (e.g. surface facing up = 0, surface facing
horizon = 90)
\end{quote}

\textbf{SurfAz} : float or DataFrame
\begin{quote}

Surface azimuth angles in decimal degrees.
SurfAz must be \textgreater{}=0 and \textless{}=360. The Azimuth convention is defined
as degrees east of north (e.g. North = 0, South=180 East = 90, West = 270).
\end{quote}

\textbf{DHI} : float or DataFrame
\begin{quote}

diffuse horizontal irradiance in W/m\textasciicircum{}2. 
DHI must be \textgreater{}=0.
\end{quote}

\textbf{DNI} : float or DataFrame
\begin{quote}

direct normal irradiance in W/m\textasciicircum{}2. 
DNI must be \textgreater{}=0.
\end{quote}

\textbf{HExtra} : float or DataFrame
\begin{quote}
\begin{description}
\item[{extraterrestrial normal irradiance in W/m\textasciicircum{}2. }] \leavevmode
HExtra must be \textgreater{}=0.

\end{description}
\end{quote}

\textbf{SunZen} : float or DataFrame
\begin{quote}

apparent (refraction-corrected) zenith
angles in decimal degrees. 
SunZen must be \textgreater{}=0 and \textless{}=180.
\end{quote}

\textbf{SunAz} : float or DataFrame
\begin{quote}

Sun azimuth angles in decimal degrees.
SunAz must be \textgreater{}=0 and \textless{}=360. The Azimuth convention is defined
as degrees east of north (e.g. North = 0, East = 90, West = 270).
\end{quote}

\textbf{AM} : float or DataFrame
\begin{quote}

relative (not pressure-corrected) airmass 
values. If AM is a DataFrame it must be of the same size as all other 
DataFrame inputs. AM must be \textgreater{}=0 (careful using the 1/sec(z) model of AM
generation)
\end{quote}

\item[{Returns}] \leavevmode
\textbf{SkyDiffuse} : float or DataFrame
\begin{quote}

the diffuse component of the solar radiation  on an
arbitrarily tilted surface defined by the Perez model as given in
reference {[}3{]}.
SkyDiffuse is the diffuse component ONLY and does not include the ground
reflected irradiance or the irradiance due to the beam.
\end{quote}

\item[{Other Parameters}] \leavevmode
\textbf{model} : string (optional, default='allsitescomposite1990')
\begin{quote}

a character string which selects the desired set of Perez
coefficients. If model is not provided as an input, the default,
`1990' will be used.
All possible model selections are:
\begin{itemize}
\item {} 
`1990'

\item {} 
`allsitescomposite1990' (same as `1990')

\item {} 
`allsitescomposite1988'

\item {} 
`sandiacomposite1988'

\item {} 
`usacomposite1988'

\item {} 
`france1988'

\item {} 
`phoenix1988'

\item {} 
`elmonte1988'

\item {} 
`osage1988'

\item {} 
`albuquerque1988'

\item {} 
`capecanaveral1988'

\item {} 
`albany1988'

\end{itemize}
\end{quote}

\end{description}\end{quote}


\strong{See also:}


{\hyperref[stubs/pvlib.pvl_ephemeris:pvlib.pvl_ephemeris]{\code{pvl\_ephemeris}}}, {\hyperref[stubs/pvlib.pvl_extraradiation:pvlib.pvl_extraradiation]{\code{pvl\_extraradiation}}}, {\hyperref[stubs/pvlib.pvl_isotropicsky:pvlib.pvl_isotropicsky]{\code{pvl\_isotropicsky}}}, {\hyperref[stubs/pvlib.pvl_haydavies1980:pvlib.pvl_haydavies1980]{\code{pvl\_haydavies1980}}}, {\hyperref[stubs/pvlib.pvl_reindl1990:pvlib.pvl_reindl1990]{\code{pvl\_reindl1990}}}, {\hyperref[stubs/pvlib.pvl_klucher1979:pvlib.pvl_klucher1979]{\code{pvl\_klucher1979}}}, {\hyperref[stubs/pvlib.pvl_kingdiffuse:pvlib.pvl_kingdiffuse]{\code{pvl\_kingdiffuse}}}, {\hyperref[stubs/pvlib.pvl_relativeairmass:pvlib.pvl_relativeairmass]{\code{pvl\_relativeairmass}}}


\paragraph{References}

{[}1{]} Loutzenhiser P.G. et. al. ``Empirical validation of models to compute
solar irradiance on inclined surfaces for building energy simulation''
2007, Solar Energy vol. 81. pp. 254-267

{[}2{]} Perez, R., Seals, R., Ineichen, P., Stewart, R., Menicucci, D., 1987. A new
simplified version of the Perez diffuse irradiance model for tilted
surfaces. Solar Energy 39(3), 221-232.

{[}3{]} Perez, R., Ineichen, P., Seals, R., Michalsky, J., Stewart, R., 1990.
Modeling daylight availability and irradiance components from direct
and global irradiance. Solar Energy 44 (5), 271-289.

{[}4{]} Perez, R. et. al 1988. ``The Development and Verification of the
Perez Diffuse Radiation Model''. SAND88-7030

\end{fulllineitems}



\section{pvlib.pvl\_haydavies1980}
\label{stubs/pvlib.pvl_haydavies1980:pvlib-pvl-haydavies1980}\label{stubs/pvlib.pvl_haydavies1980::doc}\index{pvl\_haydavies1980() (in module pvlib)}

\begin{fulllineitems}
\phantomsection\label{stubs/pvlib.pvl_haydavies1980:pvlib.pvl_haydavies1980}\pysiglinewithargsret{\code{pvlib.}\bfcode{pvl\_haydavies1980}}{\emph{SurfTilt}, \emph{SurfAz}, \emph{DHI}, \emph{DNI}, \emph{HExtra}, \emph{SunZen}, \emph{SunAz}}{}
Determine diffuse irradiance from the sky on a tilted surface using Hay \& Davies' 1980 model

Hay and Davies' 1980 model determines the diffuse irradiance from the sky
(ground reflected irradiance is not included in this algorithm) on a
tilted surface using the surface tilt angle, surface azimuth angle,
diffuse horizontal irradiance, direct normal irradiance, 
extraterrestrial irradiance, sun zenith angle, and sun azimuth angle.
\begin{quote}\begin{description}
\item[{Parameters}] \leavevmode
\textbf{SurfTilt} : float or DataFrame
\begin{quote}

Surface tilt angles in decimal degrees.
SurfTilt must be \textgreater{}=0 and \textless{}=180. The tilt angle is defined as
degrees from horizontal (e.g. surface facing up = 0, surface facing
horizon = 90)
\end{quote}

\textbf{SurfAz} : float or DataFrame
\begin{quote}

Surface azimuth angles in decimal degrees.
SurfAz must be \textgreater{}=0 and \textless{}=360. The Azimuth convention is defined
as degrees east of north (e.g. North = 0, South=180 East = 90, West = 270).
\end{quote}

\textbf{DHI} : float or DataFrame
\begin{quote}

diffuse horizontal irradiance in W/m\textasciicircum{}2. 
DHI must be \textgreater{}=0.
\end{quote}

\textbf{DNI} : float or DataFrame
\begin{quote}

direct normal irradiance in W/m\textasciicircum{}2. 
DNI must be \textgreater{}=0.
\end{quote}

\textbf{HExtra} : float or DataFrame
\begin{quote}
\begin{description}
\item[{extraterrestrial normal irradiance in W/m\textasciicircum{}2. }] \leavevmode
HExtra must be \textgreater{}=0.

\end{description}
\end{quote}

\textbf{SunZen} : float or DataFrame
\begin{quote}

apparent (refraction-corrected) zenith
angles in decimal degrees. 
SunZen must be \textgreater{}=0 and \textless{}=180.
\end{quote}

\textbf{SunAz} : float or DataFrame
\begin{quote}

Sun azimuth angles in decimal degrees.
SunAz must be \textgreater{}=0 and \textless{}=360. The Azimuth convention is defined
as degrees east of north (e.g. North = 0, East = 90, West = 270).
\end{quote}

\item[{Returns}] \leavevmode
\textbf{SkyDiffuse} : float or DataFrame
\begin{quote}

the diffuse component of the solar radiation  on an
arbitrarily tilted surface defined by the Perez model as given in
reference {[}3{]}.
SkyDiffuse is the diffuse component ONLY and does not include the ground
reflected irradiance or the irradiance due to the beam.
\end{quote}

\end{description}\end{quote}


\strong{See also:}


{\hyperref[stubs/pvlib.pvl_ephemeris:pvlib.pvl_ephemeris]{\code{pvl\_ephemeris}}}, {\hyperref[stubs/pvlib.pvl_extraradiation:pvlib.pvl_extraradiation]{\code{pvl\_extraradiation}}}, {\hyperref[stubs/pvlib.pvl_isotropicsky:pvlib.pvl_isotropicsky]{\code{pvl\_isotropicsky}}}, {\hyperref[stubs/pvlib.pvl_reindl1990:pvlib.pvl_reindl1990]{\code{pvl\_reindl1990}}}, {\hyperref[stubs/pvlib.pvl_perez:pvlib.pvl_perez]{\code{pvl\_perez}}}, {\hyperref[stubs/pvlib.pvl_klucher1979:pvlib.pvl_klucher1979]{\code{pvl\_klucher1979}}}, {\hyperref[stubs/pvlib.pvl_kingdiffuse:pvlib.pvl_kingdiffuse]{\code{pvl\_kingdiffuse}}}, {\hyperref[stubs/pvlib.pvl_spa:pvlib.pvl_spa]{\code{pvl\_spa}}}


\paragraph{References}

{[}1{]} Loutzenhiser P.G. et. al. ``Empirical validation of models to compute
solar irradiance on inclined surfaces for building energy simulation''
2007, Solar Energy vol. 81. pp. 254-267

{[}2{]} Hay, J.E., Davies, J.A., 1980. Calculations of the solar radiation incident
on an inclined surface. In: Hay, J.E., Won, T.K. (Eds.), Proc. of First
Canadian Solar Radiation Data Workshop, 59. Ministry of Supply
and Services, Canada.

\end{fulllineitems}



\section{pvlib.pvl\_isotropicsky}
\label{stubs/pvlib.pvl_isotropicsky::doc}\label{stubs/pvlib.pvl_isotropicsky:pvlib-pvl-isotropicsky}\index{pvl\_isotropicsky() (in module pvlib)}

\begin{fulllineitems}
\phantomsection\label{stubs/pvlib.pvl_isotropicsky:pvlib.pvl_isotropicsky}\pysiglinewithargsret{\code{pvlib.}\bfcode{pvl\_isotropicsky}}{\emph{SurfTilt}, \emph{DHI}}{}
Determine diffuse irradiance from the sky on a tilted surface using isotropic sky model

Hottel and Woertz's model treats the sky as a uniform source of diffuse
irradiance. Thus the diffuse irradiance from the sky (ground reflected
irradiance is not included in this algorithm) on a tilted surface can
be found from the diffuse horizontal irradiance and the tilt angle of
the surface.
\begin{quote}\begin{description}
\item[{Parameters}] \leavevmode
\textbf{SurfTilt} : float or DataFrame
\begin{quote}

Surface tilt angles in decimal degrees. 
SurfTilt must be \textgreater{}=0 and \textless{}=180. The tilt angle is defined as
degrees from horizontal (e.g. surface facing up = 0, surface facing
horizon = 90)
\end{quote}

\textbf{DHI} : float or DataFrame
\begin{quote}

Diffuse horizontal irradiance in W/m\textasciicircum{}2.
DHI must be \textgreater{}=0.
\end{quote}

\item[{Returns}] \leavevmode
\textbf{SkyDiffuse} : float of DataFrame
\begin{quote}

The diffuse component of the solar radiation  on an
arbitrarily tilted surface defined by the isotropic sky model as
given in Loutzenhiser et. al (2007) equation 3.
SkyDiffuse is the diffuse component ONLY and does not include the ground
reflected irradiance or the irradiance due to the beam.
SkyDiffuse is a column vector vector with a number of elements equal to
the input vector(s).
\end{quote}

\end{description}\end{quote}


\strong{See also:}


{\hyperref[stubs/pvlib.pvl_reindl1990:pvlib.pvl_reindl1990]{\code{pvl\_reindl1990}}}, {\hyperref[stubs/pvlib.pvl_haydavies1980:pvlib.pvl_haydavies1980]{\code{pvl\_haydavies1980}}}, {\hyperref[stubs/pvlib.pvl_perez:pvlib.pvl_perez]{\code{pvl\_perez}}}, {\hyperref[stubs/pvlib.pvl_klucher1979:pvlib.pvl_klucher1979]{\code{pvl\_klucher1979}}}, {\hyperref[stubs/pvlib.pvl_kingdiffuse:pvlib.pvl_kingdiffuse]{\code{pvl\_kingdiffuse}}}


\paragraph{References}

{[}1{]} Loutzenhiser P.G. et. al. ``Empirical validation of models to compute
solar irradiance on inclined surfaces for building energy simulation''
2007, Solar Energy vol. 81. pp. 254-267

{[}2{]} Hottel, H.C., Woertz, B.B., 1942. Evaluation of flat-plate solar heat
collector. Trans. ASME 64, 91.

\end{fulllineitems}



\section{pvlib.pvl\_kingdiffuse}
\label{stubs/pvlib.pvl_kingdiffuse::doc}\label{stubs/pvlib.pvl_kingdiffuse:pvlib-pvl-kingdiffuse}\index{pvl\_kingdiffuse() (in module pvlib)}

\begin{fulllineitems}
\phantomsection\label{stubs/pvlib.pvl_kingdiffuse:pvlib.pvl_kingdiffuse}\pysiglinewithargsret{\code{pvlib.}\bfcode{pvl\_kingdiffuse}}{\emph{SurfTilt}, \emph{DHI}, \emph{GHI}, \emph{SunZen}}{}
Determine diffuse irradiance from the sky on a tilted surface using the King model

King's model determines the diffuse irradiance from the sky
(ground reflected irradiance is not included in this algorithm) on a
tilted surface using the surface tilt angle, diffuse horizontal
irradiance, global horizontal irradiance, and sun zenith angle. Note
that this model is not well documented and has not been published in
any fashion (as of January 2012).
\begin{quote}\begin{description}
\item[{Parameters}] \leavevmode
\textbf{SurfTilt} : float or DataFrame
\begin{quote}

Surface tilt angles in decimal degrees.
SurfTilt must be \textgreater{}=0 and \textless{}=180. The tilt angle is defined as
degrees from horizontal (e.g. surface facing up = 0, surface facing
horizon = 90)
\end{quote}

\textbf{DHI} : float or DataFrame
\begin{quote}

diffuse horizontal irradiance in W/m\textasciicircum{}2. 
DHI must be \textgreater{}=0.
\end{quote}

\textbf{GHI} : float or DataFrame
\begin{quote}

global horizontal irradiance in W/m\textasciicircum{}2. 
DHI must be \textgreater{}=0.
\end{quote}

\textbf{SunZen} : float or DataFrame
\begin{quote}

apparent (refraction-corrected) zenith
angles in decimal degrees. 
SunZen must be \textgreater{}=0 and \textless{}=180.
\end{quote}

\item[{Returns}] \leavevmode
\textbf{SkyDiffuse} : float or DataFrame
\begin{quote}

the diffuse component of the solar radiation  on an
arbitrarily tilted surface as given by a model developed by David L.
King at Sandia National Laboratories.
\end{quote}

\end{description}\end{quote}


\strong{See also:}


{\hyperref[stubs/pvlib.pvl_ephemeris:pvlib.pvl_ephemeris]{\code{pvl\_ephemeris}}}, {\hyperref[stubs/pvlib.pvl_extraradiation:pvlib.pvl_extraradiation]{\code{pvl\_extraradiation}}}, {\hyperref[stubs/pvlib.pvl_isotropicsky:pvlib.pvl_isotropicsky]{\code{pvl\_isotropicsky}}}, {\hyperref[stubs/pvlib.pvl_haydavies1980:pvlib.pvl_haydavies1980]{\code{pvl\_haydavies1980}}}, {\hyperref[stubs/pvlib.pvl_perez:pvlib.pvl_perez]{\code{pvl\_perez}}}, {\hyperref[stubs/pvlib.pvl_klucher1979:pvlib.pvl_klucher1979]{\code{pvl\_klucher1979}}}, {\hyperref[stubs/pvlib.pvl_reindl1990:pvlib.pvl_reindl1990]{\code{pvl\_reindl1990}}}



\end{fulllineitems}



\section{pvlib.pvl\_klucher1979}
\label{stubs/pvlib.pvl_klucher1979:pvlib-pvl-klucher1979}\label{stubs/pvlib.pvl_klucher1979::doc}\index{pvl\_klucher1979() (in module pvlib)}

\begin{fulllineitems}
\phantomsection\label{stubs/pvlib.pvl_klucher1979:pvlib.pvl_klucher1979}\pysiglinewithargsret{\code{pvlib.}\bfcode{pvl\_klucher1979}}{\emph{SurfTilt}, \emph{SurfAz}, \emph{DHI}, \emph{GHI}, \emph{SunZen}, \emph{SunAz}}{}
Determine diffuse irradiance from the sky on a tilted surface using Klucher's 1979 model

Klucher's 1979 model determines the diffuse irradiance from the sky
(ground reflected irradiance is not included in this algorithm) on a
tilted surface using the surface tilt angle, surface azimuth angle,
diffuse horizontal irradiance, direct normal irradiance, global
horizontal irradiance, extraterrestrial irradiance, sun zenith angle,
and sun azimuth angle.
\begin{quote}\begin{description}
\item[{Parameters}] \leavevmode
\textbf{SurfTilt} : float or DataFrame
\begin{quote}

Surface tilt angles in decimal degrees.
SurfTilt must be \textgreater{}=0 and \textless{}=180. The tilt angle is defined as
degrees from horizontal (e.g. surface facing up = 0, surface facing
horizon = 90)
\end{quote}

\textbf{SurfAz} : float or DataFrame
\begin{quote}

Surface azimuth angles in decimal degrees.
SurfAz must be \textgreater{}=0 and \textless{}=360. The Azimuth convention is defined
as degrees east of north (e.g. North = 0, South=180 East = 90, West = 270).
\end{quote}

\textbf{DHI} : float or DataFrame
\begin{quote}

diffuse horizontal irradiance in W/m\textasciicircum{}2. 
DHI must be \textgreater{}=0.
\end{quote}

\textbf{GHI} : float or DataFrame
\begin{quote}

Global  irradiance in W/m\textasciicircum{}2. 
DNI must be \textgreater{}=0.
\end{quote}

\textbf{SunZen} : float or DataFrame
\begin{quote}

apparent (refraction-corrected) zenith
angles in decimal degrees. 
SunZen must be \textgreater{}=0 and \textless{}=180.
\end{quote}

\textbf{SunAz} : float or DataFrame
\begin{quote}

Sun azimuth angles in decimal degrees.
SunAz must be \textgreater{}=0 and \textless{}=360. The Azimuth convention is defined
as degrees east of north (e.g. North = 0, East = 90, West = 270).
\end{quote}

\item[{Returns}] \leavevmode
\textbf{SkyDiffuse} : float or DataFrame
\begin{quote}

the diffuse component of the solar radiation  on an
arbitrarily tilted surface defined by the Klucher model as given in
Loutzenhiser et. al (2007) equation 4.
SkyDiffuse is the diffuse component ONLY and does not include the ground
reflected irradiance or the irradiance due to the beam.
SkyDiffuse is a column vector vector with a number of elements equal to
the input vector(s).
\end{quote}

\end{description}\end{quote}


\strong{See also:}


{\hyperref[stubs/pvlib.pvl_ephemeris:pvlib.pvl_ephemeris]{\code{pvl\_ephemeris}}}, {\hyperref[stubs/pvlib.pvl_extraradiation:pvlib.pvl_extraradiation]{\code{pvl\_extraradiation}}}, {\hyperref[stubs/pvlib.pvl_isotropicsky:pvlib.pvl_isotropicsky]{\code{pvl\_isotropicsky}}}, {\hyperref[stubs/pvlib.pvl_haydavies1980:pvlib.pvl_haydavies1980]{\code{pvl\_haydavies1980}}}, {\hyperref[stubs/pvlib.pvl_perez:pvlib.pvl_perez]{\code{pvl\_perez}}}, {\hyperref[stubs/pvlib.pvl_reindl1990:pvlib.pvl_reindl1990]{\code{pvl\_reindl1990}}}, {\hyperref[stubs/pvlib.pvl_kingdiffuse:pvlib.pvl_kingdiffuse]{\code{pvl\_kingdiffuse}}}


\paragraph{References}

{[}1{]} Loutzenhiser P.G. et. al. ``Empirical validation of models to compute
solar irradiance on inclined surfaces for building energy simulation''
2007, Solar Energy vol. 81. pp. 254-267

{[}2{]} Klucher, T.M., 1979. Evaluation of models to predict insolation on tilted
surfaces. Solar Energy 23 (2), 111-114.

\end{fulllineitems}



\section{pvlib.pvl\_reindl1990}
\label{stubs/pvlib.pvl_reindl1990:pvlib-pvl-reindl1990}\label{stubs/pvlib.pvl_reindl1990::doc}\index{pvl\_reindl1990() (in module pvlib)}

\begin{fulllineitems}
\phantomsection\label{stubs/pvlib.pvl_reindl1990:pvlib.pvl_reindl1990}\pysiglinewithargsret{\code{pvlib.}\bfcode{pvl\_reindl1990}}{\emph{SurfTilt}, \emph{SurfAz}, \emph{DHI}, \emph{DNI}, \emph{GHI}, \emph{HExtra}, \emph{SunZen}, \emph{SunAz}}{}
Determine diffuse irradiance from the sky on a tilted surface using Reindl's 1990 model

Reindl's 1990 model determines the diffuse irradiance from the sky
(ground reflected irradiance is not included in this algorithm) on a
tilted surface using the surface tilt angle, surface azimuth angle,
diffuse horizontal irradiance, direct normal irradiance, global
horizontal irradiance, extraterrestrial irradiance, sun zenith angle,
and sun azimuth angle.
\begin{quote}\begin{description}
\item[{Parameters}] \leavevmode
\textbf{SurfTilt} : DataFrame
\begin{quote}

Surface tilt angles in decimal degrees.
SurfTilt must be \textgreater{}=0 and \textless{}=180. The tilt angle is defined as
degrees from horizontal (e.g. surface facing up = 0, surface facing
horizon = 90)
\end{quote}

\textbf{SurfAz} : DataFrame
\begin{quote}

Surface azimuth angles in decimal degrees.
SurfAz must be \textgreater{}=0 and \textless{}=360. The Azimuth convention is defined
as degrees east of north (e.g. North = 0, South=180 East = 90, West = 270).
\end{quote}

\textbf{DHI} : DataFrame
\begin{quote}

diffuse horizontal irradiance in W/m\textasciicircum{}2. 
DHI must be \textgreater{}=0.
\end{quote}

\textbf{DNI} : DataFrame
\begin{quote}

direct normal irradiance in W/m\textasciicircum{}2. 
DNI must be \textgreater{}=0.
\end{quote}

\textbf{GHI: DataFrame} :
\begin{quote}

Global irradiance in W/m\textasciicircum{}2. 
GHI must be \textgreater{}=0.
\end{quote}

\textbf{HExtra} : DataFrame
\begin{quote}
\begin{description}
\item[{extraterrestrial normal irradiance in W/m\textasciicircum{}2. }] \leavevmode
HExtra must be \textgreater{}=0.

\end{description}
\end{quote}

\textbf{SunZen} : DataFrame
\begin{quote}

apparent (refraction-corrected) zenith
angles in decimal degrees. 
SunZen must be \textgreater{}=0 and \textless{}=180.
\end{quote}

\textbf{SunAz} : DataFrame
\begin{quote}

Sun azimuth angles in decimal degrees.
SunAz must be \textgreater{}=0 and \textless{}=360. The Azimuth convention is defined
as degrees east of north (e.g. North = 0, East = 90, West = 270).
\end{quote}

\item[{Returns}] \leavevmode
\textbf{SkyDiffuse} : DataFrame
\begin{quote}

the diffuse component of the solar radiation  on an
arbitrarily tilted surface defined by the Reindl model as given in
Loutzenhiser et. al (2007) equation 8.
SkyDiffuse is the diffuse component ONLY and does not include the ground
reflected irradiance or the irradiance due to the beam.
SkyDiffuse is a column vector vector with a number of elements equal to
the input vector(s).
\end{quote}

\end{description}\end{quote}


\strong{See also:}


{\hyperref[stubs/pvlib.pvl_ephemeris:pvlib.pvl_ephemeris]{\code{pvl\_ephemeris}}}, {\hyperref[stubs/pvlib.pvl_extraradiation:pvlib.pvl_extraradiation]{\code{pvl\_extraradiation}}}, {\hyperref[stubs/pvlib.pvl_isotropicsky:pvlib.pvl_isotropicsky]{\code{pvl\_isotropicsky}}}, {\hyperref[stubs/pvlib.pvl_haydavies1980:pvlib.pvl_haydavies1980]{\code{pvl\_haydavies1980}}}, {\hyperref[stubs/pvlib.pvl_perez:pvlib.pvl_perez]{\code{pvl\_perez}}}, {\hyperref[stubs/pvlib.pvl_klucher1979:pvlib.pvl_klucher1979]{\code{pvl\_klucher1979}}}, {\hyperref[stubs/pvlib.pvl_kingdiffuse:pvlib.pvl_kingdiffuse]{\code{pvl\_kingdiffuse}}}


\paragraph{Notes}

The POAskydiffuse calculation is generated from the Loutzenhiser et al.
(2007) paper, equation 8. Note that I have removed the beam and ground
reflectance portion of the equation and this generates ONLY the diffuse
radiation from the sky and circumsolar, so the form of the equation
varies slightly from equation 8.
\paragraph{References}

{[}1{]} Loutzenhiser P.G. et. al. ``Empirical validation of models to compute
solar irradiance on inclined surfaces for building energy simulation''
2007, Solar Energy vol. 81. pp. 254-267

{[}2{]} Reindl, D.T., Beckmann, W.A., Duffie, J.A., 1990a. Diffuse fraction
correlations. Solar Energy 45(1), 1-7.

{[}3{]} Reindl, D.T., Beckmann, W.A., Duffie, J.A., 1990b. Evaluation of hourly
tilted surface radiation models. Solar Energy 45(1), 9-17.

\end{fulllineitems}



\chapter{Data Handling}
\label{index:data-handling}
\begin{longtable}{lp{7cm}}
\hline
\endfirsthead

\multicolumn{2}{c}%
{{\textsf{\tablename\ \thetable{} -- continued from previous page}}} \\
\hline
\endhead

\hline \multicolumn{2}{|r|}{{\textsf{Continued on next page}}} \\ \hline
\endfoot

\endlastfoot


{\hyperref[stubs/pvlib.pvl_readtmy2:pvlib.pvl_readtmy2]{\code{pvlib.pvl\_readtmy2}}}(FileName)
 & 
Read a TMY2 file in to a DataFrame
\\

{\hyperref[stubs/pvlib.pvl_readtmy3:pvlib.pvl_readtmy3]{\code{pvlib.pvl\_readtmy3}}}(FileName)
 & 
Read a TMY3 file in to a pandas dataframe
\\
\hline\end{longtable}



\section{pvlib.pvl\_readtmy2}
\label{stubs/pvlib.pvl_readtmy2::doc}\label{stubs/pvlib.pvl_readtmy2:pvlib-pvl-readtmy2}\index{pvl\_readtmy2() (in module pvlib)}

\begin{fulllineitems}
\phantomsection\label{stubs/pvlib.pvl_readtmy2:pvlib.pvl_readtmy2}\pysiglinewithargsret{\code{pvlib.}\bfcode{pvl\_readtmy2}}{\emph{FileName}}{}
Read a TMY2 file in to a DataFrame

Note that valuescontained in the DataFrame are unchanged from the TMY2 
file (i.e. units  are retained). Time/Date and Location data imported from the 
TMY2 file have been modified to a ``friendlier'' form conforming to modern
conventions (e.g. N latitude is postive, E longitude is positive, the
``24th'' hour of any day is technically the ``0th'' hour of the next day).
In the case of any discrepencies between this documentation and the 
TMY2 User's Manual ({[}1{]}), the TMY2 User's Manual takes precedence.

If a FileName is not provided, the user will be prompted to browse to
an appropriate TMY2 file.
\begin{quote}\begin{description}
\item[{Parameters}] \leavevmode
\textbf{FileName} : string
\begin{quote}

an optional argument which allows the user to select which
TMY2 format file should be read. A file path may also be necessary if
the desired TMY2 file is not in the working path. If FileName
is not provided, the user will be prompted to browse to an
appropriate TMY2 file.
\end{quote}

\item[{Returns}] \leavevmode
\textbf{TMYData} : DataFrame
\begin{quote}

A dataframe, is provided with the following components.  Note
that for more detailed descriptions of each component, please consult
the TMY2 User's Manual ({[}1{]}), especially tables 3-1 through 3-6, and 
Appendix B.
\end{quote}

\textbf{meta} : struct
\begin{quote}

A struct containing the metadata from the TMY2 file.
\end{quote}

\end{description}\end{quote}


\strong{See also:}


{\hyperref[stubs/pvlib.pvl_makelocationstruct:pvlib.pvl_makelocationstruct]{\code{pvl\_makelocationstruct}}}, \code{pvl\_maketimestruct}, {\hyperref[stubs/pvlib.pvl_readtmy3:pvlib.pvl_readtmy3]{\code{pvl\_readtmy3}}}


\paragraph{Notes}

The structures have the following fields

\begin{tabulary}{\linewidth}{|L|L|}
\hline
\textsf{\relax 
meta Field
} & \textsf{\relax }\\
\hline
meta.SiteID
 & 
Site identifier code (WBAN number), scalar unsigned integer
\\

meta.StationName
 & 
Station name, 1x1 cell string
\\

meta.StationState
 & 
Station state 2 letter designator, 1x1 cell string
\\

meta.SiteTimeZone
 & 
Hours from Greenwich, scalar double
\\

meta.latitude
 & 
Latitude in decimal degrees, scalar double
\\

meta.longitude
 & 
Longitude in decimal degrees, scalar double
\\

meta.SiteElevation
 & 
Site elevation in meters, scalar double
\\
\hline\end{tabulary}


\begin{longtable}{|l|p{11cm}|}
\hline
\textsf{\relax 
TMYData Field
} & \textsf{\relax 
Meaning
}\\
\hline\endfirsthead

\multicolumn{2}{c}%
{{\textsf{\tablename\ \thetable{} -- continued from previous page}}} \\
\hline
\textsf{\relax 
TMYData Field
} & \textsf{\relax 
Meaning
}\\
\hline\endhead

\hline \multicolumn{2}{|r|}{{\textsf{Continued on next page}}} \\ \hline
\endfoot

\endlastfoot


index
 & 
Pandas timeseries object containing timestamps
\\

year
 & \\

month
 & \\

day
 & \\

hour
 & \\

ETR
 & 
Extraterrestrial horizontal radiation recv'd during 60 minutes prior to timestamp, Wh/m\textasciicircum{}2
\\

ETRN
 & 
Extraterrestrial normal radiation recv'd during 60 minutes prior to timestamp, Wh/m\textasciicircum{}2
\\

GHI
 & 
Direct and diffuse horizontal radiation recv'd during 60 minutes prior to timestamp, Wh/m\textasciicircum{}2
\\

GHISource
 & 
See {[}1{]}, Table 3-3
\\

GHIUncertainty
 & 
See {[}1{]}, Table 3-4
\\

DNI
 & 
Amount of direct normal radiation (modeled) recv'd during 60 mintues prior to timestamp, Wh/m\textasciicircum{}2
\\

DNISource
 & 
See {[}1{]}, Table 3-3
\\

DNIUncertainty
 & 
See {[}1{]}, Table 3-4
\\

DHI
 & 
Amount of diffuse horizontal radiation recv'd during 60 minutes prior to timestamp, Wh/m\textasciicircum{}2
\\

DHISource
 & 
See {[}1{]}, Table 3-3
\\

DHIUncertainty
 & 
See {[}1{]}, Table 3-4
\\

GHillum
 & 
Avg. total horizontal illuminance recv'd during the 60 minutes prior to timestamp, units of 100 lux (e.g. value of 50 = 5000 lux)
\\

GHillumSource
 & 
See {[}1{]}, Table 3-3
\\

GHillumUncertainty
 & 
See {[}1{]}, Table 3-4
\\

DNillum
 & 
Avg. direct normal illuminance recv'd during the 60 minutes prior to timestamp, units of 100 lux
\\

DNillumSource
 & 
See {[}1{]}, Table 3-3
\\

DNillumUncertainty
 & 
See {[}1{]}, Table 3-4
\\

DHillum
 & 
Avg. horizontal diffuse illuminance recv'd during the 60 minutes prior to timestamp, units of 100 lux
\\

DHillumSource
 & 
See {[}1{]}, Table 3-3
\\

DHillumUncertainty
 & 
See {[}1{]}, Table 3-4
\\

Zenithlum
 & 
Avg. luminance at the sky's zenith during the 60 minutes prior to timestamp, units of 10 Cd/m\textasciicircum{}2 (e.g. value of 700 = 7,000 Cd/m\textasciicircum{}2)
\\

ZenithlumSource
 & 
See {[}1{]}, Table 3-3
\\

ZenithlumUncertainty
 & 
See {[}1{]}, Table 3-4
\\

TotCld
 & 
Amount of sky dome covered by clouds or obscuring phenonema at time stamp, tenths of sky
\\

TotCldSource
 & 
See {[}1{]}, Table 3-5, 8760x1 cell array of strings
\\

TotCldUnertainty
 & 
See {[}1{]}, Table 3-6
\\

OpqCld
 & 
Amount of sky dome covered by clouds or obscuring phenonema that prevent observing the sky at time stamp, tenths of sky
\\

OpqCldSource
 & 
See {[}1{]}, Table 3-5, 8760x1 cell array of strings
\\

OpqCldUncertainty
 & 
See {[}1{]}, Table 3-6
\\

DryBulb
 & 
Dry bulb temperature at the time indicated, in tenths of degree C (e.g. 352 = 35.2 C).
\\

DryBulbSource
 & 
See {[}1{]}, Table 3-5, 8760x1 cell array of strings
\\

DryBulbUncertainty
 & 
See {[}1{]}, Table 3-6
\\

DewPoint
 & 
Dew-point temperature at the time indicated, in tenths of degree C (e.g. 76 = 7.6 C).
\\

DewPointSource
 & 
See {[}1{]}, Table 3-5, 8760x1 cell array of strings
\\

DewPointUncertainty
 & 
See {[}1{]}, Table 3-6
\\

RHum
 & 
Relative humidity at the time indicated, percent
\\

RHumSource
 & 
See {[}1{]}, Table 3-5, 8760x1 cell array of strings
\\

RHumUncertainty
 & 
See {[}1{]}, Table 3-6
\\

Pressure
 & 
Station pressure at the time indicated, 1 mbar
\\

PressureSource
 & 
See {[}1{]}, Table 3-5, 8760x1 cell array of strings
\\

PressureUncertainty
 & 
See {[}1{]}, Table 3-6
\\

Wdir
 & 
Wind direction at time indicated, degrees from east of north (360 = 0 = north; 90 = East; 0 = undefined,calm)
\\

WdirSource
 & 
See {[}1{]}, Table 3-5, 8760x1 cell array of strings
\\

WdirUncertainty
 & 
See {[}1{]}, Table 3-6
\\

Wspd
 & 
Wind speed at the time indicated, in tenths of meters/second (e.g. 212 = 21.2 m/s)
\\

WspdSource
 & 
See {[}1{]}, Table 3-5, 8760x1 cell array of strings
\\

WspdUncertainty
 & 
See {[}1{]}, Table 3-6
\\

Hvis
 & 
Distance to discernable remote objects at time indicated (7777=unlimited, 9999=missing data), in tenths of kilometers (e.g. 341 = 34.1 km).
\\

HvisSource
 & 
See {[}1{]}, Table 3-5, 8760x1 cell array of strings
\\

HvisUncertainty
 & 
See {[}1{]}, Table 3-6
\\

CeilHgt
 & 
Height of cloud base above local terrain (7777=unlimited, 88888=cirroform, 99999=missing data), in meters
\\

CeilHgtSource
 & 
See {[}1{]}, Table 3-5, 8760x1 cell array of strings
\\

CeilHgtUncertainty
 & 
See {[}1{]}, Table 3-6
\\

Pwat
 & 
Total precipitable water contained in a column of unit cross section from Earth to top of atmosphere, in millimeters
\\

PwatSource
 & 
See {[}1{]}, Table 3-5, 8760x1 cell array of strings
\\

PwatUncertainty
 & 
See {[}1{]}, Table 3-6
\\

AOD
 & 
The broadband aerosol optical depth (broadband turbidity) in thousandths on the day indicated (e.g. 114 = 0.114)
\\

AODSource
 & 
See {[}1{]}, Table 3-5, 8760x1 cell array of strings
\\

AODUncertainty
 & 
See {[}1{]}, Table 3-6
\\

SnowDepth
 & 
Snow depth in centimeters on the day indicated, (999 = missing data).
\\

SnowDepthSource
 & 
See {[}1{]}, Table 3-5, 8760x1 cell array of strings
\\

SnowDepthUncertainty
 & 
See {[}1{]}, Table 3-6
\\

LastSnowfall
 & 
Number of days since last snowfall (maximum value of 88, where 88 = 88 or greater days; 99 = missing data)
\\

LastSnowfallSource
 & 
See {[}1{]}, Table 3-5, 8760x1 cell array of strings
\\

LastSnowfallUncertainty
 & 
See {[}1{]}, Table 3-6
\\

PresentWeather
 & 
See {[}1{]}, Appendix B, an 8760x1 cell array of strings. Each string contains 10 numeric values. The string can be parsed to determine each of 10 observed weather metrics.
\\
\hline\end{longtable}

\paragraph{References}

{[}1{]} Marion, W and Urban, K. ``Wilcox, S and Marion, W. ``User's Manual
for TMY2s''. NREL 1995.

\end{fulllineitems}



\section{pvlib.pvl\_readtmy3}
\label{stubs/pvlib.pvl_readtmy3:pvlib-pvl-readtmy3}\label{stubs/pvlib.pvl_readtmy3::doc}\index{pvl\_readtmy3() (in module pvlib)}

\begin{fulllineitems}
\phantomsection\label{stubs/pvlib.pvl_readtmy3:pvlib.pvl_readtmy3}\pysiglinewithargsret{\code{pvlib.}\bfcode{pvl\_readtmy3}}{\emph{FileName}}{}
Read a TMY3 file in to a pandas dataframe

Read a TMY3 file and make a pandas dataframe of the data. Note that values
contained in the struct are unchanged from the TMY3 file (i.e. units 
are retained). In the case of any discrepencies between this
documentation and the TMY3 User's Manual ({[}1{]}), the TMY3 User's Manual
takes precedence.

If a FileName is not provided, the user will be prompted to browse to
an appropriate TMY3 file.
\begin{quote}\begin{description}
\item[{Parameters}] \leavevmode
\textbf{FileName} : string
\begin{quote}

An optional argument which allows the user to select which
TMY3 format file should be read. A file path may also be necessary if
the desired TMY3 file is not in the MATLAB working path.
\end{quote}

\item[{Returns}] \leavevmode
\textbf{TMYDATA} : DataFrame
\begin{quote}

A pandas dataframe, is provided with the components in the table below. Note
that for more detailed descriptions of each component, please consult
the TMY3 User's Manual ({[}1{]}), especially tables 1-1 through 1-6.
\end{quote}

\textbf{meta} : struct
\begin{quote}

struct of meta data is created, which contains all 
site metadata available in the file
\end{quote}

\end{description}\end{quote}


\strong{See also:}


{\hyperref[stubs/pvlib.pvl_makelocationstruct:pvlib.pvl_makelocationstruct]{\code{pvl\_makelocationstruct}}}, {\hyperref[stubs/pvlib.pvl_readtmy2:pvlib.pvl_readtmy2]{\code{pvl\_readtmy2}}}


\paragraph{Notes}

\begin{tabulary}{\linewidth}{|L|L|L|}
\hline
\textsf{\relax 
meta field
} & \textsf{\relax 
format
} & \textsf{\relax 
description
}\\
\hline
meta.altitude
 & 
Float
 & 
site elevation
\\

meta.latitude
 & 
Float
 & 
site latitudeitude
\\

meta.longitude
 & 
Float
 & 
site longitudeitude
\\

meta.Name
 & 
String
 & 
site name
\\

meta.State
 & 
String
 & 
state
\\

meta.TZ
 & 
Float
 & 
timezone
\\

meta.USAF
 & 
Int
 & 
USAF identifier
\\
\hline\end{tabulary}


\begin{longtable}{{|l|p{10cm}|}}
\hline
\textsf{\relax 
TMYData field
} & \textsf{\relax 
description
}\\
\hline\endfirsthead

\multicolumn{2}{c}%
{{\textsf{\tablename\ \thetable{} -- continued from previous page}}} \\
\hline
\textsf{\relax 
TMYData field
} & \textsf{\relax 
description
}\\
\hline\endhead

\hline \multicolumn{2}{|r|}{{\textsf{Continued on next page}}} \\ \hline
\endfoot

\endlastfoot


TMYData.Index
 & 
A pandas datetime index. NOTE, the index is currently timezone unaware, and times are set to local standard time (daylight savings is not indcluded)
\\

TMYData.ETR
 & 
Extraterrestrial horizontal radiation recv'd during 60 minutes prior to timestamp, Wh/m\textasciicircum{}2
\\

TMYData.ETRN
 & 
Extraterrestrial normal radiation recv'd during 60 minutes prior to timestamp, Wh/m\textasciicircum{}2
\\

TMYData.GHI
 & 
Direct and diffuse horizontal radiation recv'd during 60 minutes prior to timestamp, Wh/m\textasciicircum{}2
\\

TMYData.GHISource
 & 
See {[}1{]}, Table 1-4
\\

TMYData.GHIUncertainty
 & 
Uncertainty based on random and bias error estimates                        see {[}2{]}
\\

TMYData.DNI
 & 
Amount of direct normal radiation (modeled) recv'd during 60 mintues prior to timestamp, Wh/m\textasciicircum{}2
\\

TMYData.DNISource
 & 
See {[}1{]}, Table 1-4
\\

TMYData.DNIUncertainty
 & 
Uncertainty based on random and bias error estimates                        see {[}2{]}
\\

TMYData.DHI
 & 
Amount of diffuse horizontal radiation recv'd during 60 minutes prior to timestamp, Wh/m\textasciicircum{}2
\\

TMYData.DHISource
 & 
See {[}1{]}, Table 1-4
\\

TMYData.DHIUncertainty
 & 
Uncertainty based on random and bias error estimates                        see {[}2{]}
\\

TMYData.GHillum
 & 
Avg. total horizontal illuminance recv'd during the 60 minutes prior to timestamp, lx
\\

TMYData.GHillumSource
 & 
See {[}1{]}, Table 1-4
\\

TMYData.GHillumUncertainty
 & 
Uncertainty based on random and bias error estimates                        see {[}2{]}
\\

TMYData.DNillum
 & 
Avg. direct normal illuminance recv'd during the 60 minutes prior to timestamp, lx
\\

TMYData.DNillumSource
 & 
See {[}1{]}, Table 1-4
\\

TMYData.DNillumUncertainty
 & 
Uncertainty based on random and bias error estimates                        see {[}2{]}
\\

TMYData.DHillum
 & 
Avg. horizontal diffuse illuminance recv'd during the 60 minutes prior to timestamp, lx
\\

TMYData.DHillumSource
 & 
See {[}1{]}, Table 1-4
\\

TMYData.DHillumUncertainty
 & 
Uncertainty based on random and bias error estimates                        see {[}2{]}
\\

TMYData.Zenithlum
 & 
Avg. luminance at the sky's zenith during the 60 minutes prior to timestamp, cd/m\textasciicircum{}2
\\

TMYData.ZenithlumSource
 & 
See {[}1{]}, Table 1-4
\\

TMYData.ZenithlumUncertainty
 & 
Uncertainty based on random and bias error estimates                        see {[}1{]} section 2.10
\\

TMYData.TotCld
 & 
Amount of sky dome covered by clouds or obscuring phenonema at time stamp, tenths of sky
\\

TMYData.TotCldSource
 & 
See {[}1{]}, Table 1-5, 8760x1 cell array of strings
\\

TMYData.TotCldUnertainty
 & 
See {[}1{]}, Table 1-6
\\

TMYData.OpqCld
 & 
Amount of sky dome covered by clouds or obscuring phenonema that prevent observing the sky at time stamp, tenths of sky
\\

TMYData.OpqCldSource
 & 
See {[}1{]}, Table 1-5, 8760x1 cell array of strings
\\

TMYData.OpqCldUncertainty
 & 
See {[}1{]}, Table 1-6
\\

TMYData.DryBulb
 & 
Dry bulb temperature at the time indicated, deg C
\\

TMYData.DryBulbSource
 & 
See {[}1{]}, Table 1-5, 8760x1 cell array of strings
\\

TMYData.DryBulbUncertainty
 & 
See {[}1{]}, Table 1-6
\\

TMYData.DewPoint
 & 
Dew-point temperature at the time indicated, deg C
\\

TMYData.DewPointSource
 & 
See {[}1{]}, Table 1-5, 8760x1 cell array of strings
\\

TMYData.DewPointUncertainty
 & 
See {[}1{]}, Table 1-6
\\

TMYData.RHum
 & 
Relatitudeive humidity at the time indicated, percent
\\

TMYData.RHumSource
 & 
See {[}1{]}, Table 1-5, 8760x1 cell array of strings
\\

TMYData.RHumUncertainty
 & 
See {[}1{]}, Table 1-6
\\

TMYData.Pressure
 & 
Station pressure at the time indicated, 1 mbar
\\

TMYData.PressureSource
 & 
See {[}1{]}, Table 1-5, 8760x1 cell array of strings
\\

TMYData.PressureUncertainty
 & 
See {[}1{]}, Table 1-6
\\

TMYData.Wdir
 & 
Wind direction at time indicated, degrees from north (360 = north; 0 = undefined,calm)
\\

TMYData.WdirSource
 & 
See {[}1{]}, Table 1-5, 8760x1 cell array of strings
\\

TMYData.WdirUncertainty
 & 
See {[}1{]}, Table 1-6
\\

TMYData.Wspd
 & 
Wind speed at the time indicated, meter/second
\\

TMYData.WspdSource
 & 
See {[}1{]}, Table 1-5, 8760x1 cell array of strings
\\

TMYData.WspdUncertainty
 & 
See {[}1{]}, Table 1-6
\\

TMYData.Hvis
 & 
Distance to discernable remote objects at time indicated (7777=unlimited), meter
\\

TMYData.HvisSource
 & 
See {[}1{]}, Table 1-5, 8760x1 cell array of strings
\\

TMYData.HvisUncertainty
 & 
See {[}1{]}, Table 1-6
\\

TMYData.CeilHgt
 & 
Height of cloud base above local terrain (7777=unlimited), meter
\\

TMYData.CeilHgtSource
 & 
See {[}1{]}, Table 1-5, 8760x1 cell array of strings
\\

TMYData.CeilHgtUncertainty
 & 
See {[}1{]}, Table 1-6
\\

TMYData.Pwat
 & 
Total precipitable water contained in a column of unit cross section from earth to top of atmosphere, cm
\\

TMYData.PwatSource
 & 
See {[}1{]}, Table 1-5, 8760x1 cell array of strings
\\

TMYData.PwatUncertainty
 & 
See {[}1{]}, Table 1-6
\\

TMYData.AOD
 & 
The broadband aerosol optical depth per unit of air mass due to extinction by aerosol component of atmosphere, unitless
\\

TMYData.AODSource
 & 
See {[}1{]}, Table 1-5, 8760x1 cell array of strings
\\

TMYData.AODUncertainty
 & 
See {[}1{]}, Table 1-6
\\

TMYData.Alb
 & 
The ratio of reflected solar irradiance to global horizontal irradiance, unitless
\\

TMYData.AlbSource
 & 
See {[}1{]}, Table 1-5, 8760x1 cell array of strings
\\

TMYData.AlbUncertainty
 & 
See {[}1{]}, Table 1-6
\\

TMYData.Lprecipdepth
 & 
The amount of liquid precipitation observed at indicated time for the period indicated in the liquid precipitation quantity field, millimeter
\\

TMYData.Lprecipquantity
 & 
The period of accumulatitudeion for the liquid precipitation depth field, hour
\\

TMYData.LprecipSource
 & 
See {[}1{]}, Table 1-5, 8760x1 cell array of strings
\\

TMYData.LprecipUncertainty
 & 
See {[}1{]}, Table 1-6
\\
\hline\end{longtable}

\paragraph{References}

{[}1{]} Wilcox, S and Marion, W. ``Users Manual for TMY3 Data Sets''.
NREL/TP-581-43156, Revised May 2008.

{[}2{]} Wilcox, S. (2007). National Solar Radiation Database 1991 2005 
Update: Users Manual. 472 pp.; NREL Report No. TP-581-41364.

\end{fulllineitems}



\chapter{System Modelling functions}
\label{index:system-modelling-functions}
\begin{longtable}{lp{7cm}}
\hline
\endfirsthead

\multicolumn{2}{c}%
{{\textsf{\tablename\ \thetable{} -- continued from previous page}}} \\
\hline
\endhead

\hline \multicolumn{2}{|r|}{{\textsf{Continued on next page}}} \\ \hline
\endfoot

\endlastfoot


{\hyperref[stubs/pvlib.pvl_physicaliam:pvlib.pvl_physicaliam]{\code{pvlib.pvl\_physicaliam}}}(K, L, n, theta)
 & 
Determine the incidence angle modifier using refractive
\\

{\hyperref[stubs/pvlib.pvl_ashraeiam:pvlib.pvl_ashraeiam]{\code{pvlib.pvl\_ashraeiam}}}(b, theta)
 & 
Determine the incidence angle modifier using the ASHRAE transmission model.
\\

{\hyperref[stubs/pvlib.pvl_calcparams_desoto:pvlib.pvl_calcparams_desoto]{\code{pvlib.pvl\_calcparams\_desoto}}}(S, Tcell, ...{[}, ...{]})
 & 
Applies the temperature and irradiance corrections to inputs for pvl\_singlediode
\\

{\hyperref[stubs/pvlib.pvl_retreiveSAM:pvlib.pvl_retreiveSAM]{\code{pvlib.pvl\_retreiveSAM}}}(name{[}, FileLoc{]})
 & 
Retreive lastest module and inverter info from SAM website
\\

{\hyperref[stubs/pvlib.pvl_sapm:pvlib.pvl_sapm]{\code{pvlib.pvl\_sapm}}}(Module, Eb, Ediff, Tcell, AM, AOI)
 & 
Performs Sandia PV Array Performance Model to get 5 points on IV curve given SAPM module parameters, Ee, and cell temperature
\\

{\hyperref[stubs/pvlib.pvl_sapmcelltemp:pvlib.pvl_sapmcelltemp]{\code{pvlib.pvl\_sapmcelltemp}}}(E, Wspd, Tamb{[}, modelt{]})
 & 
Estimate cell temperature from irradiance, windspeed, ambient temperature, and module parameters (SAPM)
\\

{\hyperref[stubs/pvlib.pvl_singlediode:pvlib.pvl_singlediode]{\code{pvlib.pvl\_singlediode}}}(Module, IL, I0, Rs, ...)
 & 
Solve the single-diode model to obtain a photovoltaic IV curve
\\

{\hyperref[stubs/pvlib.pvl_snlinverter:pvlib.pvl_snlinverter]{\code{pvlib.pvl\_snlinverter}}}(Inverter, Vmp, Pmp)
 & 
Converts DC power and voltage to AC power using Sandia's Grid-Connected PV Inverter model
\\

{\hyperref[stubs/pvlib.pvl_systemdef:pvlib.pvl_systemdef]{\code{pvlib.pvl\_systemdef}}}(TMYmeta, SurfTilt, ...)
 & 
Generates a dict of system paramters used throughout a simulation
\\
\hline\end{longtable}



\section{pvlib.pvl\_physicaliam}
\label{stubs/pvlib.pvl_physicaliam:pvlib-pvl-physicaliam}\label{stubs/pvlib.pvl_physicaliam::doc}\index{pvl\_physicaliam() (in module pvlib)}

\begin{fulllineitems}
\phantomsection\label{stubs/pvlib.pvl_physicaliam:pvlib.pvl_physicaliam}\pysiglinewithargsret{\code{pvlib.}\bfcode{pvl\_physicaliam}}{\emph{K}, \emph{L}, \emph{n}, \emph{theta}}{}
Determine the incidence angle modifier using refractive 
index, glazing thickness, and extinction coefficient

pvl\_physicaliam calculates the incidence angle modifier as described in
De Soto et al. ``Improvement and validation of a model for photovoltaic
array performance'', section 3. The calculation is based upon a physical
model of absorbtion and transmission through a cover. Required
information includes, incident angle, cover extinction coefficient,
cover thickness

Note: The authors of this function believe that eqn. 14 in {[}1{]} is
incorrect. This function uses the following equation in its place:
theta\_r = arcsin(1/n * sin(theta))
\begin{quote}\begin{description}
\item[{Parameters}] \leavevmode
\textbf{K} : float
\begin{quote}

The glazing extinction coefficient in units of 1/meters. Reference
{[}1{]} indicates that a value of  4 is reasonable for ``water white''
glass. K must be a numeric scalar or vector with all values \textgreater{}=0. If K
is a vector, it must be the same size as all other input vectors.
\end{quote}

\textbf{L} : float
\begin{quote}

The glazing thickness in units of meters. Reference {[}1{]} indicates
that 0.002 meters (2 mm) is reasonable for most glass-covered
PV panels. L must be a numeric scalar or vector with all values \textgreater{}=0. 
If L is a vector, it must be the same size as all other input vectors.
\end{quote}

\textbf{n} : float
\begin{quote}

The effective index of refraction (unitless). Reference {[}1{]}
indicates that a value of 1.526 is acceptable for glass. n must be a 
numeric scalar or vector with all values \textgreater{}=0. If n is a vector, it 
must be the same size as all other input vectors.
\end{quote}

\textbf{theta :float} :
\begin{quote}

The angle of incidence between the module normal vector and the
sun-beam vector in degrees. Theta must be a numeric scalar or vector.
For any values of theta where abs(theta)\textgreater{}90, IAM is set to 0. For any
values of theta where -90 \textless{} theta \textless{} 0, theta is set to abs(theta) and
evaluated. A warning will be generated if any(theta\textless{}0 or theta\textgreater{}90).
\end{quote}

\item[{Returns}] \leavevmode
\textbf{IAM} : float
\begin{quote}
\begin{description}
\item[{The incident angle modifier as specified in eqns. 14-16 of {[}1{]}.}] \leavevmode
IAM is a column vector with the same number of elements as the
largest input vector.

\end{description}
\end{quote}

\end{description}\end{quote}


\strong{See also:}


{\hyperref[stubs/pvlib.pvl_getaoi:pvlib.pvl_getaoi]{\code{pvl\_getaoi}}}, {\hyperref[stubs/pvlib.pvl_ephemeris:pvlib.pvl_ephemeris]{\code{pvl\_ephemeris}}}, {\hyperref[stubs/pvlib.pvl_spa:pvlib.pvl_spa]{\code{pvl\_spa}}}, {\hyperref[stubs/pvlib.pvl_ashraeiam:pvlib.pvl_ashraeiam]{\code{pvl\_ashraeiam}}}


\paragraph{References}
\begin{description}
\item[{{[}1{]} W. De Soto et al., ``Improvement and validation of a model for}] \leavevmode
photovoltaic array performance'', Solar Energy, vol 80, pp. 78-88,
2006.

\item[{{[}2{]} Duffie, John A. \& Beckman, William A.. (2006). Solar Engineering }] \leavevmode
of Thermal Processes, third edition. {[}Books24x7 version{]} Available 
from \href{http://common.books24x7.com/toc.aspx?bookid=17160}{http://common.books24x7.com/toc.aspx?bookid=17160}.

\end{description}

\end{fulllineitems}



\section{pvlib.pvl\_ashraeiam}
\label{stubs/pvlib.pvl_ashraeiam::doc}\label{stubs/pvlib.pvl_ashraeiam:pvlib-pvl-ashraeiam}\index{pvl\_ashraeiam() (in module pvlib)}

\begin{fulllineitems}
\phantomsection\label{stubs/pvlib.pvl_ashraeiam:pvlib.pvl_ashraeiam}\pysiglinewithargsret{\code{pvlib.}\bfcode{pvl\_ashraeiam}}{\emph{b}, \emph{theta}}{}
Determine the incidence angle modifier using the ASHRAE transmission model.

pvl\_ashraeiam calculates the incidence angle modifier as developed in
{[}1{]}, and adopted by ASHRAE (American Society of Heating, Refrigeration,
and Air Conditioning Engineers) {[}2{]}. The model has been used by model
programs such as PVSyst {[}3{]}.

Note: For incident angles near 90 degrees, this model has a
discontinuity which has been addressed in this function.
\begin{quote}\begin{description}
\item[{Parameters}] \leavevmode
\textbf{b} : float
\begin{quote}

A parameter to adjust the modifier as a function of angle of
incidence. Typical values are on the order of 0.05 {[}3{]}.
\end{quote}

\textbf{theta} : DataFrame
\begin{quote}

The angle of incidence between the module normal vector and the
sun-beam vector in degrees. Theta must be a numeric scalar or vector.
For any values of theta where abs(theta)\textgreater{}90, IAM is set to 0. For any
values of theta where -90 \textless{} theta \textless{} 0, theta is set to abs(theta) and
evaluated. A warning will be generated if any(theta\textless{}0 or theta\textgreater{}90).
For values of theta near 90 degrees, the ASHRAE model may be above 1
or less than 0 due to the discontinuity of secant(theta). IAM values
outside of {[}0,1{]} are set to 0 and a warning is generated.
\end{quote}

\item[{Returns}] \leavevmode
\textbf{IAM} : DataFrame
\begin{quote}

The incident angle modifier calculated as 1-b*(sec(theta)-1) as
described in {[}2,3{]}. IAM is a column vector with the same number of 
elements as the largest input vector.
\end{quote}

\end{description}\end{quote}


\strong{See also:}


{\hyperref[stubs/pvlib.pvl_getaoi:pvlib.pvl_getaoi]{\code{pvl\_getaoi}}}, {\hyperref[stubs/pvlib.pvl_ephemeris:pvlib.pvl_ephemeris]{\code{pvl\_ephemeris}}}, {\hyperref[stubs/pvlib.pvl_spa:pvlib.pvl_spa]{\code{pvl\_spa}}}, {\hyperref[stubs/pvlib.pvl_physicaliam:pvlib.pvl_physicaliam]{\code{pvl\_physicaliam}}}


\paragraph{References}

{[}1{]} Souka A.F., Safwat H.H., ``Determindation of the optimum orientations
for the double exposure flat-plate collector and its reflections''.
Solar Energy vol .10, pp 170-174. 1966.

{[}2{]} ASHRAE standard 93-77

{[}3{]} PVsyst Contextual Help. 
\href{http://files.pvsyst.com/help/index.html?iam\_loss.htm}{http://files.pvsyst.com/help/index.html?iam\_loss.htm} retrieved on
September 10, 2012

\end{fulllineitems}



\section{pvlib.pvl\_calcparams\_desoto}
\label{stubs/pvlib.pvl_calcparams_desoto:pvlib-pvl-calcparams-desoto}\label{stubs/pvlib.pvl_calcparams_desoto::doc}\index{pvl\_calcparams\_desoto() (in module pvlib)}

\begin{fulllineitems}
\phantomsection\label{stubs/pvlib.pvl_calcparams_desoto:pvlib.pvl_calcparams_desoto}\pysiglinewithargsret{\code{pvlib.}\bfcode{pvl\_calcparams\_desoto}}{\emph{S}, \emph{Tcell}, \emph{alpha\_isc}, \emph{ModuleParameters}, \emph{EgRef}, \emph{dEgdT}, \emph{M=1}, \emph{Sref=1000}, \emph{Tref=25}}{}
Applies the temperature and irradiance corrections to inputs for pvl\_singlediode
\begin{quote}

Applies the temperature and irradiance corrections to the IL, I0, 
Rs, Rsh, and a parameters at reference conditions (IL\_ref, I0\_ref,
etc.) according to the De Soto et. al description given in {[}1{]}. The
results of this correction procedure may be used in a single diode
model to determine IV curves at irradiance = S, cell temperature =
Tcell.
\end{quote}
\begin{quote}\begin{description}
\item[{Parameters}] \leavevmode
\textbf{S} : float or DataFrame
\begin{quote}

The irradiance (in W/m\textasciicircum{}2) absorbed by the module. S must be \textgreater{}= 0.
Due to a division by S in the script, any S value equal to 0 will be set to 1E-10.
\end{quote}

\textbf{Tcell} : float or DataFrame
\begin{quote}

The average cell temperature of cells within a module in C.
Tcell must be \textgreater{}= -273.15.
\end{quote}

\textbf{alpha\_isc} : float
\begin{quote}

The short-circuit current temperature coefficient of the module in units of 1/C.
\end{quote}

\textbf{ModuleParameters} : struct
\begin{quote}

parameters describing PV module performance at reference conditions according 
to DeSoto's paper. Parameters may be generated or found by lookup. For ease of use, 
PVL\_RETREIVESAM can automatically generate a struct based on the most recent SAM CEC module 
database. The ModuleParameters struct must contain (at least) the 
following 5 fields:
\begin{quote}
\begin{description}
\item[{\emph{ModuleParameters.a\_ref} - modified diode ideality factor parameter at}] \leavevmode
reference conditions (units of eV), a\_ref can be calculated from the
usual diode ideality factor (n), number of cells in series (Ns),
and cell temperature (Tcell) per equation (2) in {[}1{]}.

\item[{\emph{ModuleParameters.IL\_ref} - Light-generated current (or photocurrent) }] \leavevmode
in amperes at reference conditions. This value is referred to 
as Iph in some literature.

\item[{\emph{ModuleParameters.I0\_ref} - diode reverse saturation current in amperes, }] \leavevmode
under reference conditions.

\end{description}

\emph{ModuleParameters.Rsh\_ref} - shunt resistance under reference conditions (ohms)

\emph{ModuleParameters.Rs\_ref} - series resistance under reference conditions (ohms)
\end{quote}
\end{quote}

\textbf{EgRef} : float
\begin{quote}

The energy bandgap at reference temperature (in eV). 1.121 eV for silicon. EgRef must be \textgreater{}0.
\end{quote}

\textbf{dEgdT} : float
\begin{quote}

The temperature dependence of the energy bandgap at SRC (in 1/C).
May be either a scalar value (e.g. -0.0002677 as in {[}1{]}) or a
DataFrame of dEgdT values corresponding to each input condition (this
may be useful if dEgdT is a function of temperature).
\end{quote}

\item[{Returns}] \leavevmode
\textbf{IL} : float or DataFrame
\begin{quote}
\begin{quote}

Light-generated current in amperes at irradiance=S and 
cell temperature=Tcell.
\end{quote}
\begin{description}
\item[{I0}] \leavevmode{[}float or DataFrame {]}
Diode saturation curent in amperes at irradiance S and cell temperature Tcell.

\item[{Rs}] \leavevmode{[}float{]}
Series resistance in ohms at irradiance S and cell temperature Tcell.

\item[{Rsh}] \leavevmode{[}float or DataFrame {]}
Shunt resistance in ohms at irradiance S and cell temperature Tcell.

\item[{nNsVth}] \leavevmode{[}float or DataFrame {]}
Modified diode ideality factor at irradiance S and cell temperature
Tcell. Note that in source {[}1{]} nNsVth = a (equation 2). nNsVth is the 
product of the usual diode ideality factor (n), the number of 
series-connected cells in the module (Ns), and the thermal voltage 
of a cell in the module (Vth) at a cell temperature of Tcell.

\end{description}
\end{quote}

\item[{Other Parameters}] \leavevmode
\textbf{M} : float or DataFrame (optional, Default=1)
\begin{quote}

An optional airmass modifier, if omitted, M is given a value of 1,
which assumes absolute (pressure corrected) airmass = 1.5. In this
code, M is equal to M/Mref as described in {[}1{]} (i.e. Mref is assumed
to be 1). Source {[}1{]} suggests that an appropriate value for M
as a function absolute airmass (AMa) may be:

\begin{Verbatim}[commandchars=\\\{\}]
\PYG{g+gp}{\PYGZgt{}\PYGZgt{}\PYGZgt{} }\PYG{n}{M} \PYG{o}{=} \PYG{n}{np}\PYG{o}{.}\PYG{n}{polyval}\PYG{p}{(}\PYG{p}{[}\PYG{o}{\PYGZhy{}}\PYG{l+m+mf}{0.000126}\PYG{p}{,} \PYG{l+m+mf}{0.002816}\PYG{p}{,} \PYG{o}{\PYGZhy{}}\PYG{l+m+mf}{0.024459}\PYG{p}{,} \PYG{l+m+mf}{0.086257}\PYG{p}{,} \PYG{l+m+mf}{0.918093}\PYG{p}{]}\PYG{p}{,} \PYG{n}{AMa}\PYG{p}{)}
\end{Verbatim}

M may be a DataFrame.
\end{quote}

\textbf{Sref} : float (optional, Default=1000)
\begin{quote}

Optional reference irradiance in W/m\textasciicircum{}2. If omitted, a value of
1000 is used.
\end{quote}

\textbf{Tref} : float (Optional, Default=25)
\begin{quote}

Optional reference cell temperature in C. If omitted, a value of
25 C is used.
\end{quote}

\end{description}\end{quote}


\strong{See also:}


{\hyperref[stubs/pvlib.pvl_sapm:pvlib.pvl_sapm]{\code{pvl\_sapm}}}, {\hyperref[stubs/pvlib.pvl_sapmcelltemp:pvlib.pvl_sapmcelltemp]{\code{pvl\_sapmcelltemp}}}, {\hyperref[stubs/pvlib.pvl_singlediode:pvlib.pvl_singlediode]{\code{pvl\_singlediode}}}, \code{pvl\_retreivesam}


\paragraph{Notes}

If the reference parameters in the ModuleParameters struct are read
from a database or library of parameters (e.g. System Advisor Model),
it is important to use the same EgRef and dEgdT values that
were used to generate the reference parameters, regardless of the 
actual bandgap characteristics of the semiconductor. For example, in 
the case of the System Advisor Model library, created as described in 
{[}3{]}, EgRef and dEgdT for all modules were 1.121 and -0.0002677,
respectively.

This table of reference bandgap energies (EgRef), bandgap energy
temperature dependence (dEgdT), and ``typical'' airmass response (M) is
provided purely as reference to those who may generate their own
reference module parameters (a\_ref, IL\_ref, I0\_ref, etc.) based upon the
various PV semiconductors. Again, we stress the importance of
using identical EgRef and dEgdT when generation reference
parameters and modifying the reference parameters (for irradiance,
temperature, and airmass) per DeSoto's equations.
\begin{quote}
\begin{description}
\item[{Silicon (Si):}] \leavevmode
EgRef = 1.121
dEgdT = -0.0002677

\begin{Verbatim}[commandchars=\\\{\}]
\PYG{g+gp}{\PYGZgt{}\PYGZgt{}\PYGZgt{} }\PYG{n}{M} \PYG{o}{=} \PYG{n}{polyval}\PYG{p}{(}\PYG{p}{[}\PYG{o}{\PYGZhy{}}\PYG{l+m+mf}{0.000126} \PYG{l+m+mf}{0.002816} \PYG{o}{\PYGZhy{}}\PYG{l+m+mf}{0.024459} \PYG{l+m+mf}{0.086257} \PYG{l+m+mf}{0.918093}\PYG{p}{]}\PYG{p}{,} \PYG{n}{AMa}\PYG{p}{)}
\end{Verbatim}

Source = Reference 1

\item[{Cadmium Telluride (CdTe):}] \leavevmode
EgRef = 1.475
dEgdT = -0.0003

\begin{Verbatim}[commandchars=\\\{\}]
\PYG{g+gp}{\PYGZgt{}\PYGZgt{}\PYGZgt{} }\PYG{n}{M} \PYG{o}{=} \PYG{n}{polyval}\PYG{p}{(}\PYG{p}{[}\PYG{o}{\PYGZhy{}}\PYG{l+m+mf}{2.46E\PYGZhy{}5} \PYG{l+m+mf}{9.607E\PYGZhy{}4} \PYG{o}{\PYGZhy{}}\PYG{l+m+mf}{0.0134} \PYG{l+m+mf}{0.0716} \PYG{l+m+mf}{0.9196}\PYG{p}{]}\PYG{p}{,} \PYG{n}{AMa}\PYG{p}{)}
\end{Verbatim}

Source = Reference 4

\item[{Copper Indium diSelenide (CIS):}] \leavevmode
EgRef = 1.010
dEgdT = -0.00011

\begin{Verbatim}[commandchars=\\\{\}]
\PYG{g+gp}{\PYGZgt{}\PYGZgt{}\PYGZgt{} }\PYG{n}{M} \PYG{o}{=} \PYG{n}{polyval}\PYG{p}{(}\PYG{p}{[}\PYG{o}{\PYGZhy{}}\PYG{l+m+mf}{3.74E\PYGZhy{}5} \PYG{l+m+mf}{0.00125} \PYG{o}{\PYGZhy{}}\PYG{l+m+mf}{0.01462} \PYG{l+m+mf}{0.0718} \PYG{l+m+mf}{0.9210}\PYG{p}{]}\PYG{p}{,} \PYG{n}{AMa}\PYG{p}{)}
\end{Verbatim}

Source = Reference 4

\item[{Copper Indium Gallium diSelenide (CIGS):}] \leavevmode
EgRef = 1.15
dEgdT = ????

\begin{Verbatim}[commandchars=\\\{\}]
\PYG{g+gp}{\PYGZgt{}\PYGZgt{}\PYGZgt{} }\PYG{n}{M} \PYG{o}{=} \PYG{n}{polyval}\PYG{p}{(}\PYG{p}{[}\PYG{o}{\PYGZhy{}}\PYG{l+m+mf}{9.07E\PYGZhy{}5} \PYG{l+m+mf}{0.0022} \PYG{o}{\PYGZhy{}}\PYG{l+m+mf}{0.0202} \PYG{l+m+mf}{0.0652} \PYG{l+m+mf}{0.9417}\PYG{p}{]}\PYG{p}{,} \PYG{n}{AMa}\PYG{p}{)}
\end{Verbatim}

Source = Wikipedia

\end{description}

Gallium Arsenide (GaAs):
\begin{quote}

EgRef = 1.424
dEgdT = -0.000433
M = unknown
Source = Reference 4
\end{quote}
\end{quote}
\paragraph{References}
\begin{description}
\item[{{[}1{]} W. De Soto et al., ``Improvement and validation of a model for}] \leavevmode
photovoltaic array performance'', Solar Energy, vol 80, pp. 78-88,
2006.

\end{description}

{[}2{]} System Advisor Model web page. \href{https://sam.nrel.gov}{https://sam.nrel.gov}.
\begin{description}
\item[{{[}3{]} A. Dobos, ``An Improved Coefficient Calculator for the California}] \leavevmode
Energy Commission 6 Parameter Photovoltaic Module Model'', Journal of
Solar Energy Engineering, vol 134, 2012.

\item[{{[}4{]} O. Madelung, ``Semiconductors: Data Handbook, 3rd ed.'' ISBN}] \leavevmode
3-540-40488-0

\end{description}

\end{fulllineitems}



\section{pvlib.pvl\_retreiveSAM}
\label{stubs/pvlib.pvl_retreiveSAM:pvlib-pvl-retreivesam}\label{stubs/pvlib.pvl_retreiveSAM::doc}\index{pvl\_retreiveSAM() (in module pvlib)}

\begin{fulllineitems}
\phantomsection\label{stubs/pvlib.pvl_retreiveSAM:pvlib.pvl_retreiveSAM}\pysiglinewithargsret{\code{pvlib.}\bfcode{pvl\_retreiveSAM}}{\emph{name}, \emph{FileLoc='none'}}{}
Retreive lastest module and inverter info from SAM website

PVL\_RETREIVESAM Retreive lastest module and inverter info from SAM website.
This function will retreive either:
\begin{itemize}
\item {} 
CEC module database

\item {} 
Sandia Module database

\item {} 
Sandia Inverter database

\end{itemize}

and export it as a pandas dataframe
\begin{quote}\begin{description}
\item[{Parameters}] \leavevmode
\textbf{name: String} :
\begin{quote}

Name can be one of:
\begin{itemize}
\item {} 
`CECMod'- returns the CEC module database

\item {} 
`SandiaInverter- returns the Sandia Inverter database

\item {} 
`SandiaMod'- returns the Sandia Module database

\end{itemize}
\end{quote}

\textbf{FileLoc: String} :
\begin{quote}

Absolute path to the location of local versions of the SAM file. 
If FileLoc is specified, the latest versions of the SAM database will
not be downloaded. The selected file must be in .csv format.

If set to `select', a dialogue will open allowing the suer to navigate 
to the appropriate page.
\end{quote}

\textbf{Returns} :

\textbf{-------} :

\textbf{df: DataFrame} :
\begin{quote}

A DataFrame containing all the elements of the desired database. 
Each column representa a module or inverter, and a specific dataset
can be retreived by the command

\begin{Verbatim}[commandchars=\\\{\}]
\PYG{g+gp}{\PYGZgt{}\PYGZgt{}\PYGZgt{} }\PYG{n}{df}\PYG{o}{.}\PYG{n}{module\PYGZus{}or\PYGZus{}inverter\PYGZus{}name}
\end{Verbatim}
\end{quote}

\end{description}\end{quote}
\paragraph{Examples}

\begin{Verbatim}[commandchars=\\\{\}]
\PYG{g+gp}{\PYGZgt{}\PYGZgt{}\PYGZgt{} }\PYG{n}{Invdb}\PYG{o}{=}\PYG{n}{SAM}\PYG{o}{.}\PYG{n}{pvl\PYGZus{}retreiveSAM}\PYG{p}{(}\PYG{n}{name}\PYG{o}{=}\PYG{l+s}{\PYGZsq{}}\PYG{l+s}{SandiaInverter}\PYG{l+s}{\PYGZsq{}}\PYG{p}{)}
\PYG{g+gp}{\PYGZgt{}\PYGZgt{}\PYGZgt{} }\PYG{n}{inverter}\PYG{o}{=}\PYG{n}{Invdb}\PYG{o}{.}\PYG{n}{AE\PYGZus{}Solar\PYGZus{}Energy\PYGZus{}\PYGZus{}AE6\PYGZus{}0\PYGZus{}\PYGZus{}277V\PYGZus{}\PYGZus{}277V\PYGZus{}\PYGZus{}CEC\PYGZus{}2012\PYGZus{}}
\PYG{g+gp}{\PYGZgt{}\PYGZgt{}\PYGZgt{} }\PYG{n}{inverter}    
\PYG{g+go}{Vac           277.000000}
\PYG{g+go}{Paco         6000.000000}
\PYG{g+go}{Pdco         6165.670000}
\PYG{g+go}{Vdco          361.123000}
\PYG{g+go}{Pso            36.792300}
\PYG{g+go}{C0             \PYGZhy{}0.000002}
\PYG{g+go}{C1             \PYGZhy{}0.000047}
\PYG{g+go}{C2             \PYGZhy{}0.001861}
\PYG{g+go}{C3              0.000721}
\PYG{g+go}{Pnt             0.070000}
\PYG{g+go}{Vdcmax        600.000000}
\PYG{g+go}{Idcmax         32.000000}
\PYG{g+go}{Mppt\PYGZus{}low      200.000000}
\PYG{g+go}{Mppt\PYGZus{}high     500.000000}
\PYG{g+go}{Name: AE\PYGZus{}Solar\PYGZus{}Energy\PYGZus{}\PYGZus{}AE6\PYGZus{}0\PYGZus{}\PYGZus{}277V\PYGZus{}\PYGZus{}277V\PYGZus{}\PYGZus{}CEC\PYGZus{}2012\PYGZus{}, dtype: float64}
\end{Verbatim}

\end{fulllineitems}



\section{pvlib.pvl\_sapm}
\label{stubs/pvlib.pvl_sapm:pvlib-pvl-sapm}\label{stubs/pvlib.pvl_sapm::doc}\index{pvl\_sapm() (in module pvlib)}

\begin{fulllineitems}
\phantomsection\label{stubs/pvlib.pvl_sapm:pvlib.pvl_sapm}\pysiglinewithargsret{\code{pvlib.}\bfcode{pvl\_sapm}}{\emph{Module}, \emph{Eb}, \emph{Ediff}, \emph{Tcell}, \emph{AM}, \emph{AOI}}{}
Performs Sandia PV Array Performance Model to get 5 points on IV curve given SAPM module parameters, Ee, and cell temperature

The Sandia PV Array Performance Model (SAPM) generates 5 points on a PV
module's I-V curve (Voc, Isc, Ix, Ixx, Vmp/Imp) according to
SAND2004-3535. Assumes a reference cell temperature of 25 C.
\begin{quote}\begin{description}
\item[{Parameters}] \leavevmode
\textbf{Module} : DataFrame
\begin{quote}

A DataFrame defining the SAPM performance parameters (see
pvl\_retreivesam)
\end{quote}

\textbf{Ee} : float of DataFrame
\begin{quote}

The effective irradiance incident upon the module (suns). Any Ee\textless{}0
are set to 0.
\end{quote}

\textbf{celltemp} : float of DataFrame
\begin{quote}

The cell temperature (degrees C)
\end{quote}

\item[{Returns}] \leavevmode
\textbf{Result - DataFrame} :
\begin{quote}

A DataFrame with:
\begin{itemize}
\item {} 
Result.Isc

\item {} 
Result.Imp

\item {} 
Result.Ix

\item {} 
Result.Ixx

\item {} 
Result.Voc

\item {} 
Result.Vmp

\item {} 
Result.Pmp

\end{itemize}
\end{quote}

\end{description}\end{quote}


\strong{See also:}


\code{pvl\_retreivesam}, {\hyperref[stubs/pvlib.pvl_sapmcelltemp:pvlib.pvl_sapmcelltemp]{\code{pvl\_sapmcelltemp}}}


\paragraph{Notes}

The particular coefficients from SAPM which are required in Module
are:

\begin{tabulary}{\linewidth}{|L|L|}
\hline
\textsf{\relax 
Module field
} & \textsf{\relax 
Description
}\\
\hline
Module.c
 & 
1x8 vector with the C coefficients Module.c(1) = C0
\\

Module.Isc0
 & 
Short circuit current at reference condition (amps)
\\

Module.Imp0
 & 
Maximum power current at reference condition (amps)
\\

Module.AlphaIsc
 & 
Short circuit current temperature coefficient at reference condition (1/C)
\\

Module.AlphaImp
 & 
Maximum power current temperature coefficient at reference condition (1/C)
\\

Module.BetaVoc
 & 
Open circuit voltage temperature coefficient at reference condition (V/C)
\\

Module.mBetaVoc
 & 
Coefficient providing the irradiance dependence for the BetaVoc temperature coefficient at reference irradiance (V/C)
\\

Module.BetaVmp
 & 
Maximum power voltage temperature coefficient at reference condition
\\

Module.mBetaVmp
 & 
Coefficient providing the irradiance dependence for the BetaVmp temperature coefficient at reference irradiance (V/C)
\\

Module.n
 & 
Empirically determined ``diode factor'' (dimensionless)
\\

Module.Ns
 & 
Number of cells in series in a module's cell string(s)
\\
\hline\end{tabulary}

\paragraph{References}

{[}1{]} King, D. et al, 2004, ``Sandia Photovoltaic Array Performance Model'', SAND Report
3535, Sandia National Laboratories, Albuquerque, NM

\end{fulllineitems}



\section{pvlib.pvl\_sapmcelltemp}
\label{stubs/pvlib.pvl_sapmcelltemp:pvlib-pvl-sapmcelltemp}\label{stubs/pvlib.pvl_sapmcelltemp::doc}\index{pvl\_sapmcelltemp() (in module pvlib)}

\begin{fulllineitems}
\phantomsection\label{stubs/pvlib.pvl_sapmcelltemp:pvlib.pvl_sapmcelltemp}\pysiglinewithargsret{\code{pvlib.}\bfcode{pvl\_sapmcelltemp}}{\emph{E}, \emph{Wspd}, \emph{Tamb}, \emph{modelt='Open\_rack\_cell\_glassback'}, \emph{**kwargs}}{}
Estimate cell temperature from irradiance, windspeed, ambient temperature, and module parameters (SAPM)

Estimate cell and module temperatures per the Sandia PV Array
Performance model (SAPM, SAND2004-3535), when given the incident
irradiance, wind speed, ambient temperature, and SAPM module
parameters.
\begin{quote}\begin{description}
\item[{Parameters}] \leavevmode
\textbf{E} : float or DataFrame
\begin{quote}

Total incident irradiance in W/m\textasciicircum{}2. Must be \textgreater{}=0.
\end{quote}

\textbf{windspeed} : float or DataFrame
\begin{quote}

Wind speed in m/s at a height of 10 meters. Must be \textgreater{}=0
\end{quote}

\textbf{Tamb} : float or DataFrame
\begin{quote}

Ambient dry bulb temperature in degrees C. Must be \textgreater{}= -273.15.
\end{quote}

\item[{Returns}] \leavevmode
\textbf{Tcell} : float or DataFrame
\begin{quote}

Cell temperatures in degrees C.
\end{quote}

\textbf{Tmodule} : float or DataFrame
\begin{quote}

Module back temperature in degrees C.
\end{quote}

\item[{Other Parameters}] \leavevmode
\textbf{modelt} :  string

\textbf{Model to be used for parameters, can be:} :
\begin{itemize}
\item {} 
`Open\_rack\_cell\_glassback' (DEFAULT)

\item {} 
`Roof\_mount\_cell\_glassback'

\item {} 
`Open\_rack\_cell\_polymerback'

\item {} 
`Insulated\_back\_polumerback'

\item {} 
`Open\_rack\_Polymer\_thinfilm\_steel'

\item {} 
`22X\_Concentrator\_tracker'

\end{itemize}

\textbf{a} : float (optional)
\begin{quote}

SAPM module parameter for establishing the upper limit for module 
temperature at low wind speeds and high solar irradiance (see SAPM
eqn. 11). Must be a scalar.If not input, this value will be taken from the chosen
model
\end{quote}

\textbf{b} : float (optional)
\begin{quote}

SAPM module parameter for establishing the rate at which the module
temperature drops as wind speed increases (see SAPM eqn. 11). Must be
a scalar.If not input, this value will be taken from the chosen
model
\end{quote}

\textbf{deltaT} : float (optional)
\begin{quote}

SAPM module parameter giving the temperature difference
between the cell and module back surface at the reference irradiance,
E0. Must be a numeric scalar \textgreater{}=0. If not input, this value will be taken from the chosen
model
\end{quote}

\end{description}\end{quote}


\strong{See also:}


{\hyperref[stubs/pvlib.pvl_sapm:pvlib.pvl_sapm]{\code{pvl\_sapm}}}


\paragraph{References}

{[}1{]} King, D. et al, 2004, ``Sandia Photovoltaic Array Performance Model'', SAND Report
3535, Sandia National Laboratories, Albuquerque, NM

\end{fulllineitems}



\section{pvlib.pvl\_singlediode}
\label{stubs/pvlib.pvl_singlediode::doc}\label{stubs/pvlib.pvl_singlediode:pvlib-pvl-singlediode}\index{pvl\_singlediode() (in module pvlib)}

\begin{fulllineitems}
\phantomsection\label{stubs/pvlib.pvl_singlediode:pvlib.pvl_singlediode}\pysiglinewithargsret{\code{pvlib.}\bfcode{pvl\_singlediode}}{\emph{Module}, \emph{IL}, \emph{I0}, \emph{Rs}, \emph{Rsh}, \emph{nNsVth}, \emph{**kwargs}}{}
Solve the single-diode model to obtain a photovoltaic IV curve

pvl\_singlediode solves the single diode equation {[}1{]}:
I = IL - I0*{[}exp((V+I*Rs)/(nNsVth))-1{]} - (V + I*Rs)/Rsh
for I and V when given IL, I0, Rs, Rsh, and nNsVth (nNsVth = n*Ns*Vth) which
are described later. pvl\_singlediode returns a struct which contains
the 5 points on the I-V curve specified in SAND2004-3535 {[}3{]}. 
If all IL, I0, Rs, Rsh, and nNsVth are scalar, a single curve
will be returned, if any are DataFrames (of the same length), multiple IV
curves will be calculated.
\begin{quote}\begin{description}
\item[{Parameters}] \leavevmode
\textbf{These imput parameters can be calculated using PVL\_CALCPARAMS\_DESOTO from} :

\textbf{meterological data.} :

\textbf{IL} : float or DataFrame
\begin{quote}

Light-generated current (photocurrent) in amperes under desired IV
curve conditions.
\end{quote}

\textbf{I0} : float or DataFrame
\begin{quote}

Diode saturation current in amperes under desired IV curve
conditions.
\end{quote}

\textbf{Rs} : float or DataFrame
\begin{quote}

Series resistance in ohms under desired IV curve conditions.
\end{quote}

\textbf{Rsh} : float or DataFrame
\begin{quote}

Shunt resistance in ohms under desired IV curve conditions. May
be a scalar or DataFrame, but DataFrames must be of same length as all
other input DataFrames.
\end{quote}

\textbf{nNsVth} : float or DataFrame
\begin{quote}

the product of three components. 1) The usual diode ideal 
factor (n), 2) the number of cells in series (Ns), and 3) the cell 
thermal voltage under the desired IV curve conditions (Vth).
The thermal voltage of the cell (in volts) may be calculated as 
k*Tcell/q, where k is Boltzmann's constant (J/K), Tcell is the
temperature of the p-n junction in Kelvin, and q is the elementary 
charge of an electron (coulombs).
\end{quote}

\textbf{Other Parameters} :

\textbf{----------------} :

\textbf{NumPoints} : integer
\begin{quote}

Number of points in the desired IV curve (optional). Must be a finite 
scalar value. Non-integer values will be rounded to the next highest
integer (ceil). If ceil(NumPoints) is \textless{} 2, no IV curves will be produced
(i.e. Result.V and Result.I will not be generated). The default
value is 0, resulting in no calculation of IV points other than
those specified in {[}3{]}.
\end{quote}

\textbf{Returns} :

\textbf{Result} : DataFrame
\begin{quote}

A DataFrame with the following fields. All fields have the
same number of rows as the largest input DataFrame:
\begin{itemize}
\item {} 
Result.Isc -  short circuit current in amperes.

\item {} 
Result.Voc -  open circuit voltage in volts.

\item {} 
Result.Imp -  current at maximum power point in amperes.

\item {} 
Result.Vmp -  voltage at maximum power point in volts.

\item {} 
Result.Pmp -  power at maximum power point in watts.

\item {} 
Result.Ix -  current, in amperes, at V = 0.5*Voc.

\item {} 
Result.Ixx -  current, in amperes, at V = 0.5*(Voc+Vmp).

\end{itemize}
\end{quote}

\end{description}\end{quote}


\strong{See also:}


{\hyperref[stubs/pvlib.pvl_sapm:pvlib.pvl_sapm]{\code{pvl\_sapm}}}, {\hyperref[stubs/pvlib.pvl_calcparams_desoto:pvlib.pvl_calcparams_desoto]{\code{pvl\_calcparams\_desoto}}}


\paragraph{Notes}

The solution employed to solve the implicit diode equation utilizes
the Lambert W function to obtain an explicit function of V=f(i) and
I=f(V) as shown in {[}2{]}.
\paragraph{References}

{[}1{]} S.R. Wenham, M.A. Green, M.E. Watt, ``Applied Photovoltaics'' 
ISBN 0 86758 909 4

{[}2{]} A. Jain, A. Kapoor, ``Exact analytical solutions of the parameters of 
real solar cells using Lambert W-function'', Solar Energy Materials 
and Solar Cells, 81 (2004) 269-277.

{[}3{]} D. King et al, ``Sandia Photovoltaic Array Performance Model'',
SAND2004-3535, Sandia National Laboratories, Albuquerque, NM

\end{fulllineitems}



\section{pvlib.pvl\_snlinverter}
\label{stubs/pvlib.pvl_snlinverter:pvlib-pvl-snlinverter}\label{stubs/pvlib.pvl_snlinverter::doc}\index{pvl\_snlinverter() (in module pvlib)}

\begin{fulllineitems}
\phantomsection\label{stubs/pvlib.pvl_snlinverter:pvlib.pvl_snlinverter}\pysiglinewithargsret{\code{pvlib.}\bfcode{pvl\_snlinverter}}{\emph{Inverter}, \emph{Vmp}, \emph{Pmp}}{}
Converts DC power and voltage to AC power using Sandia's Grid-Connected PV Inverter model

Determine the AC power output of an inverter given the DC voltage, DC
power, and appropriate Sandia Grid-Connected Photovoltaic Inverter
Model parameters. The output, ACPower, is clipped at the maximum power
output, and gives a negative power during low-input power conditions,
but does NOT account for maximum power point tracking voltage windows
nor maximum current or voltage limits on the inverter.
\begin{quote}\begin{description}
\item[{Parameters}] \leavevmode
\textbf{Inverter} : DataFrame
\begin{quote}

A DataFrame defining the inverter to be used, giving the
inverter performance parameters according to the Sandia
Grid-Connected Photovoltaic Inverter Model (SAND 2007-5036) {[}1{]}. A set of
inverter performance parameters are provided with PV\_LIB, or may be
generated from a System Advisor Model (SAM) {[}2{]} library using pvl\_retreivesam.
\begin{quote}

Required DataFrame components are:
\end{quote}

\begin{tabulary}{\linewidth}{|L|L|}
\hline
\textsf{\relax 
Field
} & \textsf{\relax 
DataFrame
}\\
\hline
Inverter.Pac0
 & 
AC-power output from inverter based on input power and voltage, (W)
\\

Inverter.Pdc0
 & 
DC-power input to inverter, typically assumed to be equal to the PV array maximum power, (W)
\\

Inverter.Vdc0
 & 
DC-voltage level at which the AC-power rating is achieved at the reference operating condition, (V)
\\

Inverter.Ps0
 & 
DC-power required to start the inversion process, or self-consumption by inverter, strongly influences inverter efficiency at low power levels, (W)
\\

Inverter.C0
 & 
Parameter defining the curvature (parabolic) of the relationship between ac-power and dc-power at the reference operating condition, default value of zero gives a linear relationship, (1/W)
\\

Inverter.C1
 & 
Empirical coefficient allowing Pdco to vary linearly with dc-voltage input, default value is zero, (1/V)
\\

Inverter.C2
 & 
empirical coefficient allowing Pso to vary linearly with dc-voltage input, default value is zero, (1/V)
\\

Inverter.C3
 & 
empirical coefficient allowing Co to vary linearly with dc-voltage input, default value is zero, (1/V)
\\

Inverter.Pnt
 & 
ac-power consumed by inverter at night (night tare) to maintain circuitry required to sense PV array voltage, (W)
\\
\hline\end{tabulary}

\end{quote}

\textbf{Vdc} : float or DataFrame
\begin{quote}

DC voltages, in volts, which are provided as input to the inverter. Vdc must be \textgreater{}= 0.
\end{quote}

\textbf{Pdc} : float or DataFrame
\begin{quote}
\begin{description}
\item[{A scalar or DataFrame of DC powers, in watts, which are provided}] \leavevmode
as input to the inverter. Pdc must be \textgreater{}= 0.

\end{description}
\end{quote}

\item[{Returns}] \leavevmode
\textbf{ACPower} : float or DataFrame
\begin{quote}

Mdeled AC power output given the input 
DC voltage, Vdc, and input DC power, Pdc. When ACPower would be 
greater than Pac0, it is set to Pac0 to represent inverter 
``clipping''. When ACPower would be less than Ps0 (startup power
required), then ACPower is set to -1*abs(Pnt) to represent nightly 
power losses. ACPower is not adjusted for maximum power point
tracking (MPPT) voltage windows or maximum current limits of the
inverter.
\end{quote}

\end{description}\end{quote}


\strong{See also:}


{\hyperref[stubs/pvlib.pvl_sapm:pvlib.pvl_sapm]{\code{pvl\_sapm}}}, \code{pvl\_samlibrary}, {\hyperref[stubs/pvlib.pvl_singlediode:pvlib.pvl_singlediode]{\code{pvl\_singlediode}}}


\paragraph{References}

{[}1{]} (SAND2007-5036, ``Performance Model for Grid-Connected Photovoltaic 
Inverters by D. King, S. Gonzalez, G. Galbraith, W. Boyson)

{[}2{]} System Advisor Model web page. \href{https://sam.nrel.gov}{https://sam.nrel.gov}.

\end{fulllineitems}



\section{pvlib.pvl\_systemdef}
\label{stubs/pvlib.pvl_systemdef::doc}\label{stubs/pvlib.pvl_systemdef:pvlib-pvl-systemdef}\index{pvl\_systemdef() (in module pvlib)}

\begin{fulllineitems}
\phantomsection\label{stubs/pvlib.pvl_systemdef:pvlib.pvl_systemdef}\pysiglinewithargsret{\code{pvlib.}\bfcode{pvl\_systemdef}}{\emph{TMYmeta}, \emph{SurfTilt}, \emph{SurfAz}, \emph{Albedo}, \emph{SeriesModules}, \emph{ParallelModules}}{}
Generates a dict of system paramters used throughout a simulation
\begin{quote}\begin{description}
\item[{Parameters}] \leavevmode
\textbf{TMYmeta} : struct or dict
\begin{quote}
\begin{quote}

meta file generated from a TMY file using pvl\_readtmy2 or pvl\_readtmy3.
It should contain at least the following fields:
\begin{quote}

\begin{tabulary}{\linewidth}{|L|L|L|}
\hline
\textsf{\relax 
meta field
} & \textsf{\relax 
format
} & \textsf{\relax 
description
}\\
\hline
meta.altitude
 & 
Float
 & 
site elevation
\\

meta.latitude
 & 
Float
 & 
site latitude
\\

meta.longitude
 & 
Float
 & 
site longitude
\\

meta.Name
 & 
String
 & 
site name
\\

meta.State
 & 
String
 & 
state
\\

meta.TZ
 & 
Float
 & 
timezone
\\
\hline\end{tabulary}

\end{quote}
\end{quote}
\begin{description}
\item[{SurfTilt}] \leavevmode{[}float or DataFrame{]}
Surface tilt angles in decimal degrees.
SurfTilt must be \textgreater{}=0 and \textless{}=180. The tilt angle is defined as
degrees from horizontal (e.g. surface facing up = 0, surface facing
horizon = 90)

\item[{SurfAz}] \leavevmode{[}float or DataFrame{]}
Surface azimuth angles in decimal degrees.
SurfAz must be \textgreater{}=0 and \textless{}=360. The Azimuth convention is defined
as degrees east of north (e.g. North = 0, South=180 East = 90, West = 270).

\item[{Albedo}] \leavevmode{[}float or DataFrame {]}
Ground reflectance, typically 0.1-0.4 for
surfaces on Earth (land), may increase over snow, ice, etc. May also 
be known as the reflection coefficient. Must be \textgreater{}=0 and \textless{}=1.

\item[{SeriesModules}] \leavevmode{[}float{]}
Number of modules connected in series in a string.

\item[{ParallelModules}] \leavevmode{[}int{]}
Number of strings connected in parallel.

\end{description}
\end{quote}

\item[{Returns}] \leavevmode
\textbf{Result} : dict
\begin{quote}

A dict with the following fields.
\begin{itemize}
\item {} 
`SurfTilt'

\item {} 
`SurfAz'

\item {} 
`Albedo'

\item {} 
`SeriesModules'

\item {} 
`ParallelModules'

\item {} 
`Lat'

\item {} 
`Long'

\item {} 
`TZ'

\item {} 
`name'

\item {} 
`altitude'

\end{itemize}
\end{quote}

\end{description}\end{quote}


\strong{See also:}


{\hyperref[stubs/pvlib.pvl_readtmy3:pvlib.pvl_readtmy3]{\code{pvl\_readtmy3}}}, {\hyperref[stubs/pvlib.pvl_readtmy2:pvlib.pvl_readtmy2]{\code{pvl\_readtmy2}}}



\end{fulllineitems}



\chapter{PVLIB functions}
\label{index:pvlib-functions}
\begin{longtable}{lp{7cm}}
\hline
\endfirsthead

\multicolumn{2}{c}%
{{\textsf{\tablename\ \thetable{} -- continued from previous page}}} \\
\hline
\endhead

\hline \multicolumn{2}{|r|}{{\textsf{Continued on next page}}} \\ \hline
\endfoot

\endlastfoot


{\hyperref[stubs/pvlib.pvl_tools.Parse:pvlib.pvl_tools.Parse]{\code{pvlib.pvl\_tools.Parse}}}(dct, Expect)
 & 
Parses inputs to pvlib\_python functions.
\\

{\hyperref[stubs/pvlib.pvl_tools.repack:pvlib.pvl_tools.repack]{\code{pvlib.pvl\_tools.repack}}}(dct)
 & 
Converts a dict to a struct
\\

{\hyperref[stubs/pvlib.pvl_tools.cosd:pvlib.pvl_tools.cosd]{\code{pvlib.pvl\_tools.cosd}}}(angle)
 & 
Cosine with angle input in degrees
\\

{\hyperref[stubs/pvlib.pvl_tools.sind:pvlib.pvl_tools.sind]{\code{pvlib.pvl\_tools.sind}}}(angle)
 & 
Sine with angle input in degrees
\\
\hline\end{longtable}



\section{pvlib.pvl\_tools.Parse}
\label{stubs/pvlib.pvl_tools.Parse:pvlib-pvl-tools-parse}\label{stubs/pvlib.pvl_tools.Parse::doc}\index{Parse (class in pvlib.pvl\_tools)}

\begin{fulllineitems}
\phantomsection\label{stubs/pvlib.pvl_tools.Parse:pvlib.pvl_tools.Parse}\pysiglinewithargsret{\strong{class }\code{pvlib.pvl\_tools.}\bfcode{Parse}}{\emph{dct}, \emph{Expect}}{}
Parses inputs to pvlib\_python functions.
\begin{quote}\begin{description}
\item[{Parameters}] \leavevmode
\textbf{kwargs} : dict
\begin{quote}

Input variables
\end{quote}

\textbf{Expect} : dict
\begin{quote}

Parsing logic for input variables, in the form of a dict.

Possible flags are:

\begin{tabulary}{\linewidth}{|L|L|}
\hline
\textsf{\relax 
string flag
} & \textsf{\relax 
Description
}\\
\hline
num
 & 
Ensure input is numeric
\\

array
 & 
Ensure input is a numpy array. If it is not in an array, it will attempt to convert it
\\

df
 & 
Ensure input is a pandas dataframe
\\

str
 & 
Ensure inpus is a string
\\

optional
 & 
Input is not required
\\

default
 & 
Defines a default value for a parameter. Must be followed by a `default=x' statement
\\

\emph{logical}
 & 
Can accept a range of logical arguments in the form `x{[}== \textless{}= \textgreater{}= \textless{} \textgreater{}{]}value'
\\
\hline\end{tabulary}

\end{quote}

\item[{Returns}] \leavevmode
\textbf{var} : struct
\begin{quote}

Structure containing all input values in kwargs
\end{quote}

\end{description}\end{quote}
\paragraph{Notes}

This function will raise a descriptive exception if a variable fails any input requrements
\paragraph{Methods}

\begin{longtable}{lp{7cm}}
\hline
\endfirsthead

\multicolumn{2}{c}%
{{\textsf{\tablename\ \thetable{} -- continued from previous page}}} \\
\hline
\endhead

\hline \multicolumn{2}{|r|}{{\textsf{Continued on next page}}} \\ \hline
\endfoot

\endlastfoot


\code{parse\_fcn}(kwargs, Expect)
 & 

\\
\hline\end{longtable}

\index{\_\_init\_\_() (pvlib.pvl\_tools.Parse method)}

\begin{fulllineitems}
\phantomsection\label{stubs/pvlib.pvl_tools.Parse:pvlib.pvl_tools.Parse.__init__}\pysiglinewithargsret{\bfcode{\_\_init\_\_}}{\emph{dct}, \emph{Expect}}{}
\end{fulllineitems}

\paragraph{Methods}

\begin{longtable}{lp{7cm}}
\hline
\endfirsthead

\multicolumn{2}{c}%
{{\textsf{\tablename\ \thetable{} -- continued from previous page}}} \\
\hline
\endhead

\hline \multicolumn{2}{|r|}{{\textsf{Continued on next page}}} \\ \hline
\endfoot

\endlastfoot


{\hyperref[stubs/pvlib.pvl_tools.Parse:pvlib.pvl_tools.Parse.__init__]{\code{\_\_init\_\_}}}(dct, Expect)
 & 

\\

\code{parse\_fcn}(kwargs, Expect)
 & 

\\
\hline\end{longtable}


\end{fulllineitems}



\section{pvlib.pvl\_tools.repack}
\label{stubs/pvlib.pvl_tools.repack:pvlib-pvl-tools-repack}\label{stubs/pvlib.pvl_tools.repack::doc}\index{repack (class in pvlib.pvl\_tools)}

\begin{fulllineitems}
\phantomsection\label{stubs/pvlib.pvl_tools.repack:pvlib.pvl_tools.repack}\pysiglinewithargsret{\strong{class }\code{pvlib.pvl\_tools.}\bfcode{repack}}{\emph{dct}}{}
Converts a dict to a struct
\begin{quote}\begin{description}
\item[{Parameters}] \leavevmode
\textbf{dct} : dict
\begin{quote}

Dict to be converted
\end{quote}

\item[{Returns}] \leavevmode
\textbf{self} : struct
\begin{quote}

structure of packed dict entries
\end{quote}

\end{description}\end{quote}
\index{\_\_init\_\_() (pvlib.pvl\_tools.repack method)}

\begin{fulllineitems}
\phantomsection\label{stubs/pvlib.pvl_tools.repack:pvlib.pvl_tools.repack.__init__}\pysiglinewithargsret{\bfcode{\_\_init\_\_}}{\emph{dct}}{}
\end{fulllineitems}

\paragraph{Methods}

\begin{longtable}{lp{7cm}}
\hline
\endfirsthead

\multicolumn{2}{c}%
{{\textsf{\tablename\ \thetable{} -- continued from previous page}}} \\
\hline
\endhead

\hline \multicolumn{2}{|r|}{{\textsf{Continued on next page}}} \\ \hline
\endfoot

\endlastfoot


{\hyperref[stubs/pvlib.pvl_tools.repack:pvlib.pvl_tools.repack.__init__]{\code{\_\_init\_\_}}}(dct)
 & 

\\
\hline\end{longtable}


\end{fulllineitems}



\section{pvlib.pvl\_tools.cosd}
\label{stubs/pvlib.pvl_tools.cosd::doc}\label{stubs/pvlib.pvl_tools.cosd:pvlib-pvl-tools-cosd}\index{cosd() (in module pvlib.pvl\_tools)}

\begin{fulllineitems}
\phantomsection\label{stubs/pvlib.pvl_tools.cosd:pvlib.pvl_tools.cosd}\pysiglinewithargsret{\code{pvlib.pvl\_tools.}\bfcode{cosd}}{\emph{angle}}{}
Cosine with angle input in degrees
\begin{quote}\begin{description}
\item[{Parameters}] \leavevmode
\textbf{angle} : float
\begin{quote}

Angle in degrees
\end{quote}

\item[{Returns}] \leavevmode
\textbf{result} : float
\begin{quote}

Cosine of the angle
\end{quote}

\end{description}\end{quote}

\end{fulllineitems}



\section{pvlib.pvl\_tools.sind}
\label{stubs/pvlib.pvl_tools.sind:pvlib-pvl-tools-sind}\label{stubs/pvlib.pvl_tools.sind::doc}\index{sind() (in module pvlib.pvl\_tools)}

\begin{fulllineitems}
\phantomsection\label{stubs/pvlib.pvl_tools.sind:pvlib.pvl_tools.sind}\pysiglinewithargsret{\code{pvlib.pvl\_tools.}\bfcode{sind}}{\emph{angle}}{}
Sine with angle input in degrees
\begin{quote}\begin{description}
\item[{Parameters}] \leavevmode
\textbf{angle} : float
\begin{quote}

Angle in degrees
\end{quote}

\item[{Returns}] \leavevmode
\textbf{result} : float
\begin{quote}

Sin of the angle
\end{quote}

\end{description}\end{quote}

\end{fulllineitems}



\chapter{Indices and tables}
\label{index:indices-and-tables}\begin{itemize}
\item {} 
\emph{genindex}

\item {} 
\emph{modindex}

\item {} 
\emph{search}

\end{itemize}



\renewcommand{\indexname}{Index}
\printindex
\end{document}
